\documentclass[11pt]{article}
\usepackage[utf8]{inputenc}
\usepackage[T1]{fontenc}
\pagestyle{plain}
\usepackage{times,geometry,graphicx,color,floatflt,setspace}
\usepackage[hyphens]{url}
\usepackage{wrapfig}
\geometry{margin=2.5cm}
%\onehalfspacing
\newenvironment{narrow}[2]{% 
 \begin{list}{}{% 
  \setlength{\topsep}{0pt}% 
  \setlength{\leftmargin}{#1}% 
  \setlength{\rightmargin}{#2}% 
  \setlength{\listparindent}{\parindent}% 
  \setlength{\itemindent}{\parindent}% 
  \setlength{\parsep}{\parskip}% 
 }% 
\item[]}{\end{list}} 
% \begin{narrow}{1.0cm}{1.0cm}


\begin{document}
\begin{center}
\includegraphics{cover.png}
\end{center}
\begin{flushright}
Tim Love (\texttt{tlove136{@}ntlworld.com})\\
\today
\end{flushright}
\newpage
\section*{Introduction}
These are the notes of some workshops that I've given. They are not the script
of the talks - they give rather more information than I gave on the day.

 ``Time and Narration'' has no exercises. Others  have integral exercises or a few exercises tagged onto the end.  The final workshop is nothing but exercises.


\tableofcontents


\newpage
\section{Story Beginnings and Endings }
\subsection*{Beginnings}
\textit{"It was a dark and stormy night ..."}



The opening of a short story is more important than a novel's - it's a bigger proportion of the whole, and because a story's supposed to have a tight structure, the beginning's a strong indicator of the whole story's genre, mood and tone. Sometimes they even suggest what the end's going to be. Here's the first paragraph of a story. How's it going to end?

\begin{narrow}{1.0cm}{1.0cm}
"You're not going out with him and that's the end of it!" Jenny's father
announced.
\end{narrow}

(I'll show you the ending later. For now here's a clue - it's from "Yours", a Women's magazine).




Beginnings often set the scene in some way. In the olden days, first paragraphs were info-dumps. Here's something from 1859.



\begin{narrow}{1.0cm}{1.0cm}
As Mr. John Oakhurst, gambler, stepped into the main street of Poker Flat on the morning of the twenty-third of November, 1850, he was conscious of a change in its moral atmosphere since the preceding night. Two or three men, conversing
earnestly together, ceased as he approached, and exchanged significant glances. There was a Sabbath lull in the air which, in a settlement unused to Sabbath influences, looked ominous.
\end{narrow}

Initial info-dumps aren't necessarily bad - beginnings like "Gregory Samsa woke from uneasy dreams one morning to find himself changed into a giant bug" quickly give the reader the required context - but newer stories tend to spread out the scene-setting, sometimes starting in the middle of the action - medias in res. A survey came up with these statistics for how stories begin: 40\% "narrative", 30\% "description" (e.g. info-dumps), 10\% "speech", 5\% "author comment".




In "More five-minute writing", Margret Geraghty lists a few common types of starts


\begin{itemize}
\item Attention-grabbing - "The strangest thing about my wife's return from the dead was how other people reacted" ("The Beginner's Goodbye", Anne Tyler)

\item Visceral - mentioning a strong smell, taste, etc

\item A Question - "Was was it, Tiffany Aching wondered, that people liked noise so much?" ("I shall wear midnight", Terry Pratchett)

\item An aphorism - "All happy families are alike; each unhappy family is unhappy in its own way." ("Anna Karenina", Leo Tolstoy)
\end{itemize}

\subsection*{Endings}
\textit{"Then I woke up and realised it was all a dream."}



According to "The Narrative Modes" by D.S. Brewer (where I also got the statistics from) "Endings are even more various and harder to classify. They are also apparently harder to write well". Here are the statistics for endings: 31\% speech, 10\% ironic (in novels the percentage is lower), 8\% main character dies, 7\% a symbolic final event (a door closing, a journey ends, etc), 5\% a question, 4\% "author comment", 1\% wedding (in novels the percentage is far higher). Over 15\% end with a sentence of 5 words or less.



Aristotle thought that endings should be "inevitable and unexpected". William Goldman wrote that "The key to all story endings is to give the audience what it wants, but not the way it expects".
After Poe, surprise endings became popular and influential - Though surprise endings as not ... numerically dominant in the whole of any writer's work until O.Henry, the effect of the surprise endings on short-story structure and on the popularity of the form extended beyond the actual number of examples". The importance of the ending can be so strong that it affects the shape of the whole story. Even authors who don't exclusively use twist-endings may be very end-oriented in their writing procedures - "If I didn't know the ending of a story, I wouldn't begin. I always write my last lines, my last paragraph, my last page first, and then I go back and work towards it" Katherine Anne Porter (in Writers at Work: "The Paris Review Interviews", p.151)




As an example of how closely the ending and beginning can be tied together, here's the end of the story from "Yours"

\begin{narrow}{1.0cm}{1.0cm}
Mrs Wilson winked at her daughter and said: "So he's not such a bad catch after all!"
\end{narrow}

Sometimes the ending refers back to the beginning even more explicitly. Here's the start and end of "The Joy Luck Club" by Amy Tan



\begin{itemize}
\item  My father has asked me to be the fourth corner of the Joy Luck Club. I am to replace my mother, whose seat at the mah jong table has been empty since she died two months ago.
\item And I am sitting at my mother's place at the mah jong table, on the East, where things begin.
\end{itemize}

More often, the beginning prefigures the end - symbolically maybe, a broken dish fore-shadowing the infidelity revealed in the punch-line. I think it's always worth reading the beginning again after reaching the end of a story. Here's the start of "Words from a Glass Bubble" by Vanessa Gebbie - an award winning modern story. 

\begin{narrow}{1.0cm}{1.0cm}
The Virgin Mary spoke to Eva Duffy from a glass bubble in a niche half way 
up the stairs. Eva, the post woman, heard the Virgin's words in her stomach 
more than in her ears, and she called her the VM. The VM didn't seem to mind. 
She was plastic, six inches high, hand painted, and appeared to be growing out 
of a mass of very green foliage and very pink flowers, more suited to a fish 
tank. She held a naked Infant Jesus who stretched his arms out to Eva and 
mouthed, every so often ... "Carry?"
\end{narrow}

And here's the final paragraph. It has many echoes of the first - checklist the first paragraph's items to see what happened to them. Identify themes.

\begin{narrow}{1.0cm}{1.0cm}
Then, there was a sound. The cry of a buzzard as it might have been made by 
a small boy, a thin little cry that rose triumphant into the post woman's 
house, echoed round the stairs and floated out of the open windows to 
disappear among the whispers of wind in the night sky
\end{narrow}

Here's the first paragraph of Steve Almond's "Donkey Greedy, Donkey Gets Punched", the first story of Best American Short Stories 2010. It's a typical well-crafted opening - informative about the main character and establishing the narrative voice.

\begin{narrow}{1.0cm}{1.0cm}
Dr. Raymond Oss had become, in the restless leisure of his late middle age, a 
poker player. He had a weakness for the game and the ruthless depressives it 
attracted, one of which it probably was, fair enough, though it wasn't 
something he wanted known. Oss was a psychoanalyst in private practice and
 the head of two committees at the San Francisco Institute. He was a short 
man with a meticulous Trotsky beard and a flair for hats that did not suit 
him. He cured souls, very expensively, from an office near his home in 
Redwood City
\end{narrow}



\subsection*{Open Endings - and Beginnings}

As the 20th century progressed, the trend was not to end with an explicit authorial comment (which is one reason why stories more often end in speech nowadays), and not to link the start too tightly to the ending. Increasingly, endings are "open" rather than "closed".




William Gibson's "Neuromancer" ends with a section entitled "Coda: Departure and Arrival" which sums up what many endings do - some mysteries are solved, but others are begun.




Here are some typical modern endings which give a feeling of closure but leave some doors open - new relationship, new realisations, new starts.



\begin{itemize}
    \item And that was what she remembered. That was what she always said to Queenie later, how all the future had come flooding in with him, through the open door.
    \item Later, the plane makes a slow circle over New York City, and on it two men hold hands, eyes closed, and breathe in unison.
    \item Gazing up into the darkness I saw myself as a creature driven and derided by vanity; and my eyes burned with anguish and anger.
    \item After a while I went out and left the hospital and walked back to the hotel in the rain (one of 47 ending that  Hemingway tried for "A Farewell to Arms")

\end{itemize}


Just as some authors chopped the traditional beginning from a story, so some chopped off the ending, but even stories (by Barthelme, Coover, etc) without conventional closure exploit "closure signals" (repetition, change of tone or voice, zoom-out) to prepare the reader for the end.




Nowadays the expectation of closure is so low that the climax of a text can be the moment when the theme or genre is revealed. Beginnings are becoming more ambiguous too. Here are some initial paragraphs. Trying to guess the genre is hard enough let alone guessing the ending




\begin{itemize}
\item Once a mouse family lived under the floor of a playroom. There was a mother mouse and a father mouse. There was a big sister mouse called Mousikin and a baby brother mouse called Little Mouse.
    \item Once upon a time there was a little old woman, who lived in her council flat, and was as lonely as lonely could be. She had been retired from her old job at Superdrug when her hearing seemed to be on the wane. She had accepted her glass clock meekly, and the last-day paper cup of fizzy wine, then cried on the bus all the way home.
    \item So Pete Crocker, the sheriff of Barnstable County, which was the whole of Cape Cod, came into the Federal Ethical Suicide Parlor in Hyannis one May afternoon - and he told the two six-foot hostesses there that they weren't to be alarmed, but that a notorious nothinghead named Billy the Poet was believed headed for the Cape.
    \item  We defended the city as best we could. The arrows of the Comanches came in clouds. The war clubs of the Comanches clattered on the soft, yellow pavements. There were earthworks along the Boulevard Mark Clark and the hedges had been laced with sparkling wire. People were trying to understand. I spoke to Sylvia. 'Do you think this is a good life?' The table held apples, books, long-playing records. She looked up. 'No.'
    \item I'm Push the bully, and what I hate are new kids and sissies, dumb kids and smart, rich kids, poor kids, kids who wear glasses, talk funny, show off, patrol boys and wise guys and kids who pass pencils and water the plants - and cripples, especially cripples. I love nobody loved.
\end{itemize}

The shorter the fiction, the more significant the ending becomes. In the Guardian (14/5/12) David Gaffney wrote ``Make sure the ending isn't at the end.
In micro-fiction there's a danger that much of the engagement with the story takes place when the reader has stopped reading. To avoid this, place the denouement in the middle of the story, allowing us time, as the rest of the text spins out, to consider the situation along with the narrator, and ruminate on the decisions his characters have taken. If you're not careful, micro-stories can lean towards punchline-based or "pull back to reveal" endings which have a one-note, gag-a-minute feel - the drum roll and cymbal crash. Avoid this by giving us almost all the information we need in the first few lines, using the next few paragraphs to take us on a journey below the surface.''




Short stories are usually too short to offer alternative endings in the way that "The French Lieutenant's Women" (Fowles) and "The Black Prince" (Murdoch) do.



\subsection*{See Also}
\begin{itemize}
\item Stanley Fish, in "To Write A Sentence: And How To Read One", discusses first and last sentences. He likes Roth's "When I first saw Brenda she asked me to hold her glasses" (the start of "Goodbye Columbus") because it quickly introduces the personalities. He likes Fitzgerald's "So we beat on, boats against the current, borne back ceaselessly into the past." (the end of "The Great Gatsby") because the narrative goes forwards but the sounds repeat.
\item \url{http://blogs.chi.ac.uk/shortstoryforum/listen-come-in-here-you-want-to-know-about-this/} (Listen. Come in Here. You Want to Know About This) by Victoria Heath
\end{itemize}

\subsection*{Exercises}
Match the beginnings to the endings
\subsubsection*{Beginnings}
\begin{enumerate}
\item As Mr. John Oakhurst, gambler, stepped into the main street of Poker Flat on the morning of the twenty-third of November, 1850, he was conscious of a change in its moral atmosphere since the preceding night. Two or three men, conversing earnestly together, ceased as he approached, and exchanged significant glances. There was a Sabbath lull in the air which, in a settlement unused to Sabbath influences, looked ominous.
\item Once a mouse family lived under the floor of a playroom. There was a 
mother mouse and a father mouse. There was a big sister mouse called Mousikin
and a baby brother mouse called Little Mouse.


\item ``You're not going out with him and that's the end of it!'' Jenny's
father announced.


\item I'd been following my brother's trail for two weeks now, and I knew I was
close. I never let on to them that I knew, Mum and Pete or the coppers and
social services. They had no idea what he was into. But I found out, I
figured it all out. He told me, see. Left me clues.

\item Mrs Magee had always liked a chat. Even when there was no-one to talk
to, she often spoke out loud, especially if she was cross about something.

\item It was two weeks after her mother's funeral. They gave her back her summer
job at the telephone exchange. So, on the night of the Lynmouth disaster
she was there, trying to put people through as the river rose up in the
darkness and plucked the houses away from the village. At the end of the
night shift the girls stood in a group, crying. But, Patricia didn't want
that. She pushed her bike home through the moist darkness, her arms
trembling.


\item Once upon a time there was a little old woman, who lived in her council
flat, and was as lonely as lonely could be. She had been retired from her
old job at Superdrug when her hearing seemed to be on the wane. She had
accepted her glass clock meekly, and the last-day paper cup of fizzy wine,
then cried on the bus all the way home.

\item Herrie was adjusting the nutrient feeders in the
hydroponics vats when the brainwave came. It was only his second 'wave so
he immediately left his job and headed for the Chapel. The other robots
must have received the 'wave too, for Herrie soon became one of many
heading in that direction.

\item ``Well, Mabel, and what are you going to do with yourself?" asked Joe, with
foolish flippancy. He felt quite safe himself. Without listening for an
answer, he turned aside, worked a grain of tobacco to the tip of his tongue
and spat it out. He did not care about anything, since he felt safe
himself.

\item So Pete Crocker, the sheriff of Barnstable County, which was the whole of
Cape Cod, came into the Federal Ethical Suicide Parlor in Hyannis one May
afternoon - and he told the two six-foot hostesses there that they weren't
to be alarmed, but that a notorious nothinghead named Billy the Poet was
believed headed for the Cape.

\item I'm Push the bully, and what I hate are new kids and sissies, dumb kids and
smart, rich kids, poor kids, kids who wear glasses, talk funny, show off,
patrol boys and wise guys and kids who pass pencils and water the plants -
and cripples, \textit{especially} cripples. I love nobody loved.

\item We defended the city as best we could. The arrows of the Comanches came in
clouds. The war clubs of the Comanches clattered on the soft, yellow
pavements. There were earthworks along the Boulevard Mark Clark and the
hedges had been laced with sparkling wire. People were trying to
understand. I spoke to Sylvia. 'Do you think this is a good life?' The
table held apples, books, long-playing records. She looked up. 'No.'

\item Neil's mother, Mrs Campbell, sits on her lawn chair behind a card table
outside the food co-op. Every few minutes, as the sun shifts, she moves the
chair and table several inches back so as to remain in the shade. It is a
hundred degrees outside, and bright white. Each time someone goes in or out
of the co-op a gust of air-conditioning flies out of the automatic doors,
raising dust from the cement.

\item There were two brothers, Pete and Donald.

\item Wally Czerbiak was sane. At least he was as sane as Monty Vitalle, who stood behind the second of three chairs at The Clean Cut barbershop, five dollars for a haircut, three barbers, no waiting. But sometimes there were only two barbers, sometimes only one. Actually, there was usually only one, Monty. It didn't matter. There was never much of a wait, if any, at The Clean Cut.

\item My father has asked me to be the fourth corner of the Joy Luck Club. I am
to replace my mother, whose seat at the mah jong table has been empty since
she died two months ago. My father thinks she was killed by her own
thoughts.

\item Ron stared up from his bed, through the gap torn in their corrugated iron
roof by the last 'big wind', to the stars glimmering in the night sky
beyond. It was cold tonight, more so than usual. Ice glittered on the bare
metal walls of their one roomed hut; his breath fogged on the air.

\item Once there was a little boy called Sam. He said to his mother, ``I am
lonely. Where can I find a friend?''
\item North Richmond Street being blind, was a quiet street except at the hour when the Christian Brothers' School set the boys free. An uninhabited house of two storeys stood at the blind end, detached from its neighbours in a square ground The other houses of the street, conscious of decent lives within them, gazed at one another with brown imperturbable faces. 
\end{enumerate}

\subsubsection*{Endings}
The author and genre are included as hints.
\begin{enumerate}
\item And pulseless and cold, with a Derringer by his side and a bullet in his heart, though still calm as in life, beneath the snow lay he who was at once the strongest and yet the weakest of the outcasts of Poker Flat. - "The Outcasts of Poker Flat''  (Bret Harte). \textit{Literature (1869)}
\item ``I am so happy I feel as if all the world was cheese,'' he said. And
so he was. - ``The Adventures of Little Mouse'' (Margaret Mahy). \textit{Children's}

\item Mrs Wilson winked at her daughter and said: ``So he's not such a bad catch after all!'' - ``It's only rock'n'roll'' (Yours, issue 062).  \textit{Women's}

\item It'll be somewhere up high, where you can see the whole Valley. Above all
the houses, and roads, and pylons. And when I find him I'll tell him I'm
sorry. - ``Joyriders'', Sarah Oswald  (Riptide 2). \textit{Literature}

\item Charlie gazes at her with great admiration. And Mrs. Magee knows that 
one day soon - unlike the answering service or the cash machine - he will talk back to her. And that will be the start of a very long chat indeed. - ``Talk to me please!'' (Yours, issue 041). \textit{Women's} 

\item And that was what she remembered. That was what she always said to Queenie
later, how all the future had come flooding in with him, through the open
door. - ``Around us, the Dark'', Liz Gifford (Riptide 3). \textit{Literature}

\item Mama, said the little thing at last, I am nearly dead. Mama, he said, I
love you. And the poor seamstress cried and apologised and loved the poor
wee thing until he starved to death at last. Everyone said after that, that
she was a queer old fish, carting around that funny little rag doll, all
wrapped up in scraps of lace and satin. - ``Love'', Padrika Tarrant (Broken Things). \textit{Literature}

\item ``Walk," he said. And it did, not noticing the spongy encrustations that it
crushed beneath its feet. - ``Kismet'', Keith Brooke (Dream). \textit{SF}

\item ``No, I want you, I want you," was all he answered, blindly, with the
terrible intonation which frightened her almost more than her horror lest
he should \textit{not} want her. - ``The Horse Dealer's Daughter'', D.H. Lawrence. \textit{Literature}

\item When Nancy raised her eyes at last to the book and bottle, she saw that
there was a label on the bottle. What the label said was this: WELCOME TO
THE MONKEY HOUSE. - ``Welcome to the Monkey House'', Kurt Vonnegut. \textit{Literature/SF}


\item I can't stand them near me. I move against them. I shove them away. I force
them off. I press them, thrust them aside. \textit{I push through}. - ``A Poetics for Bullies'',  Stanley Elkin. \textit{Literature}

\item We killed a great many in the south suddenly with helicopters and rockets
but we found that those we had killed were children and more came from the
north and from the east and from other places where there are children
preparing to live. 'Skin,' Miss R. said softly in the white, yellow room.
'This is the Clemency Committee. And would you remove your belt and
shoelaces.' I removed my belt and shoelaces and looked (rain shattering
from a great height the prospects of silence and clear, near rows of houses
in the sub-divisions) into their savage black eyes, paint, feathers, beads. - ``The Indian Uprising'', Donald Barthelme. \textit{Literature}

\item Later, the plane makes a slow circle over New York City, and on it two men
hold hands, eyes closed, and breathe in unison. - ``Territory'', David Leavitt.
 \textit{Literature}

\item And in this way, smiling, nodding to the music, he went another mile or so
and pretended that he was not already slowing down, that he was not going
to turn back, that he would be able to drive on like this, alone, and have
the right answer when his wife stood before him in the doorway of his home
and asked, Where is he? Where is your brother? - ``The Rich Brother'', Tobias Wolff. \textit{Literature}

\item It was then that Wally Czerbiak decided not to die. And he didn't. - ``The Shooting of John Roy Worth'', Stuart M Kounsky (Ellery Queen's Mystery Magazine). \textit{Mystery}


\item And I am sitting at my mother's place at the mah jong table, on the East,
where things begin. - ``The Joy Luck Club'', Amy Tan. \textit{Literature}


\item Somewhere in the fields a starship's engines began to warm, the low hum
growing into a full throaty roar. Ron began to laugh aloud, the first for a
very long time, as the ship spat towards the stars. He knew that his chance
had come and that quite soon he too would be on his way. - ``Living in the starship's shadow'', Mark Iles (Dream). \textit{SF}

\item And this story is called ``The Boy Who Went Looking for a Friend'', and here
is an end to it. - ``The Boy Who Went Looking for a Friend'' (Margaret Mahy). \textit{Children's} 
\item Gazing up into the darkness I saw myself as a creature driven and derided by vanity; and my eyes burned with anguish and anger. - ``Araby'' (James Joyce).  \textit{Literature}
\end{enumerate}



\newpage
\section{Dialogue for writers }
\subsection*{Real Dialogue}
Before looking at dialogue in fiction, let's consider real dialogue in more detail. Though speeches and debating skills have been researched for millennia, research into conversation began only a few decades  ago. Here's an example from "Language and Creativity: the art of common talk" by Ronald Carter (Routledge, 2004)

\begin{narrow}{1.0cm}{1.0cm}
A: \textit{Yes, he must have a bob or two.}

B:  \textit{Whatever he does he makes money out of it, just like that.}

C:  \textit{Bob's your uncle.}

B:  \textit{He's quite a lot of money erm tied up in property and things. 
   He's got a finger in all kinds of pies and houses and stuff.}
\end{narrow}
It's banter. Some information is exchanged but quite a lot of other things are happening too.

Conversation can also expose the pecking order of the participants. Some of them interrupt, some affect the direction of the discussion. We have a fair idea of how people should behave in certain contexts, even as children. Here's another example (from "Conversation Analysis and Discourse Analysis" by Robin Wooffitt (Sage, 2005))

\begin{narrow}{1.0cm}{1.0cm}
Child:  \textit{Have to cut these Mummy} [pause 1s]

Child:  \textit{Won't we Mummy} [pause 1s]

Child:  \textit{Won't we}
Mother: Yes
\end{narrow}

\subsubsection*{The Rules}
There are patterns and expectations in conversation that we notice especially when they're not obeyed.

We know how to take turns, anticipating when the speaker will stop. Pauses and ends of sentences are good places to interrupt. People who don't want to be interrupted avoid pausing at the end of sentences. Turn-taking follows various conventions. If you get them mixed up, you'll be interpreted as shy or rude. 



We know when to ask open questions and when to target questions at particular people. We recognise controversial statements and deliberate attempts to disrupt conversational norms. We use a range of techniques, but mostly we're expected to abide by a few principles (known as Grice's Maxims, etc). Briefly they're that we say the right amount and quality of relevant words in an appropriate fashion. Any deviations from these maxims are potentially suspicious and revealing.

 

In "The Organised Mind" (Penguin, 2015), Daniel Levitin suggests a situation where 2 equally-ranked but competitive office workers are in a hot room. The one further from the window might not say "Open the window", but they might say "Gosh, it's getting warm here?" How should the workmate respond? Should the reply show that they understand the game, or should they break the rules?


\subsection*{Literary Uses of Dialogue}
As we've seen, conversation has many purposes in real life, not all of which are replicated in stories (though it's possible in film). In prose, dialogue has more literary uses -


\begin{itemize}
\item Show not tell
\item Return the narrative to "real-time" after a passage of summarising text
\item Reveal personality (after a few lines you can know a lot about someone)
\item Add variety of texture - breaks up blocks of description 
\item Change of Point-of-View (not always easy to do otherwise)
\item Advance plot rapidly (characters can jump and summarise in a way that narrators can't always get away with)
\item Flashbacks and Info-dumps
\item  Increase dramatic tension (especially when one character knows something that another doesn't)
\item Flexibility - Characters can lie, say outrageous things and make grammatical errors. Narrators can't do this so easily

\end{itemize}

\subsection*{Realism}
How real should the dialogue be? As we've already seen, in real life there's redundancy, hesitation, mistakes, etc - all the things we're told not to do when writing. How many of these can we get away with in dialogue? The odd "Um" or "well" is surely ok. Ungrammatical phrases are ok (indeed, we'd expect some characters not to speak the Queen's English). But these effects can become tedious if over-used.



One common issue is whether speech should be rendered phonetically? How about this?

\begin{narrow}{1.0cm}{1.0cm}
 \textit{Too many bastards ken ma Montgomery Street address.  Cash oan the nail! Partin 
wi that poppy wis the hardest bit. The easiest wis ma last shot, taken in ma 
left airm this morning.  Ah needed something tae keep us gaun during this 
period ay intense preparation.  Then ah wis off like a rocket roond the
 Kirkgate, whizzing through ma shopping list.} ("Trainspotting", Irvine Welsh)
\end{narrow}

What are your views on that?


\begin{itemize}
\item "Dialects are awkward to convey properly in print, and always look very hammy when the author attempts to write them down phonetically in the cause of accuracy. It's far better to leave them to the readers' imagination, and just indicate by the occasional phrase or regional word ... a little dialect goes a long way in fiction"  (Jean Saunders, "Writing Dialogue - The Essential Guide", p.119)
\item "If writing dialogue for a character with a specific accent, don't write it out phonetically, as this can look patronizing and old-fashioned. Use odd syntax and a few choice bits of slang to convey their accent." (Rowena Macdonald)
\item Adam Sexton (in "Master Class in Fiction Writing", McGraw-Hill) considers phonetic writing as discrimitating against certain types. It often assumes that the default reader uses received pronunciation.

\end{itemize}

And what about historical fiction? Emma Darwin points out that "you're not forging documents, you're writing fiction for readers now: you're after evocation and perhaps verisimiltude, definitely not pastiche and reproduction".

\subsection*{Exercise 1}
Convert at least some of this to dialogue.
\begin{narrow}{1.0cm}{1.0cm}
 \textit{The whole platoon crept stealthily through the bushes, pausing to listen every now and then for the
slightest sound of branches snapping or leaves rustling. The situation was taut,
 strained, and each man
had his own gutful of fear, and his own private thoughts of Hell in this steaming jungle. The whole idea
was to take the enemy by surprise, but the constant feeling of being watched mad
e them certain that
they were as much the hunted as the hunters.}
\end{narrow}

\subsection*{Exercise 2}
Add more narrative and tags to the following, removing as much dialogue as
 you want (Start of ``I'll take New York'' by Miranda Dickinson).
\begin{narrow}{1.0cm}{1.0cm}
'Bea?'

\textit{Five more minutes ...}

'Bea, honey, why don't we just order? I don't think he's ...'

'He's definitely not ...'

'\textit{Shh!} Can't you \textit{see} she's upset?'

'What? I'm just saying ...'

\textit{He'll} be \textit{here. I know he will ...}

'I think he stood her up.'

'Could you say that any louder? Only I don't think the waiter in the restaurant 
across the \textit{street} heard you ...'
\end{narrow}

Improve the following
\begin{narrow}{1.0cm}{1.0cm}
``You can't mean it,'' she exclaimed.

``I assure you, I mean every word,'' he smirked.

``Oh, you're too, too cruel,'' she moaned.

``You better believe it, babe,'' he sneered.

Improve the following

'Very well,' conceded Williamson reluctantly. 'But you are paying.'

'I cannot be long,' warned Chaloner, supposing there was no harm in listening. H
e might learn something useful with no obligation to reciprocate. 'I have an aud
ience with the Queen.'

'And you say you have no connections,' said Lester wonderingly.


\end{narrow}
 Which version would you prefer?

\subsection*{When to use dialogue}
There are competitions for stories that are completely dialogue. Dave Eggers’s "Your Fathers, Where Are They? And The Prophets, Do They Live For Ever?", Philip Roth's "Deception" and Nicholson Baker's "Checkpoint" are all or mostly dialogue. A.B. Yehoshua's "Mr. Mani" is dialogue where we only see one person's words! Usually though, dialogue's used more sparingly. Proust for example didn't use it much.



You'll often find speech at the start or end of a story. Someone worked out that
10\% of stories begin with "speech", and 31\% end with it. However starting with dialogue might be a risky option nowadays
\begin{itemize}
\item  opening a story with dialogue "was popular at the turn of the last century; it looks musty now. The problem with beginning a story with dialogue is that the reader knows absolutely nothing about the first character to appear in a story. … That requires that she read on a bit further to make sense of the dialogue. Then, at least briefly, she has to kind of backtrack in her mind to put it all into context. That represents, at the least, a speed bump, and at worst, a complete stall." (Les Edgerton, "Hooked: Write Fiction That Grabs Readers At Page One")

\item 
"It’s a typical pet peeve of editors and agents: Stories that begin with dialogue." (Jane Friedman) 

\item "beginning a novel with dialogue is hard. It's very difficult to do it effectively, because the reader doesn't have context, they don't yet know why they should care, and a lot of people are turned off by gratuitous in media res. … If you can pull it off, fantastic, if not, an agent will be able to tell very quickly" (Nathan Bransford)
\end{itemize}

Dialogue is often used at pivotal emotional moments- "John, I don't love you any more" is fast and effective. Often it's a character who (without realising) states the story's main conflict or moral. 



Dialogue is sometimes thought to be inherently more lively and interesting than plain narrative (it's considered Action rather than Narrative), so some writers use it to replace back-story, info-dumps, introspection, etc. It often fails, lapsing into monologue.



\subsection*{Tags}
What are they for? They tell the reader who's speaking and how they say it. But they're a common source of complaint. What about this? 

\begin{narrow}{1.0cm}{1.0cm}
   \textit{"No!" he snarled angrily, his eyes full of suspicion.}
\end{narrow}

Writing manuals often say that most of the time it's best to use “he said” or “she said” (it's more or less invisible) or nothing at all. The following isn't a good idea.


\begin{narrow}{1.0cm}{1.0cm}
"You can’t mean it," she exclaimed.

"I assure you, I mean every word," he smirked.

"Oh, you’re too, too cruel," she moaned.

"You better believe it, babe," he sneered.
\end{narrow}
If one's writing with a restricted point-of-view, one needs to be especially careful with adverbs, because they express far more than the intonation and eyes that they're describing can express - "darkly", "hopefully"



\subsubsection*{Identification of speakers}
One can use body language instead of tags, thus avoiding the "Talking Heads" risk. i.e. instead of
\begin{narrow}{1.0cm}{1.0cm}
 \textit{"Our fence needs mending", John said.} 
\end{narrow}
use

\begin{narrow}{1.0cm}{1.0cm}
 \textit{John looked out of the window. "Our fence needs mending."}
\end{narrow}

\subsubsection*{Intonation}
Beware of adverbs. "boastfully", "flirtingly", "humourously, "justifiably" are surely redundant. Instead you may need to work harder at the phrasing to compensate for the loss of intonation - if you want to add emphasis to the final word of "I'll go to the shops tomorrow" you could use "I'll go to the shops \textit{tomorrow}", "I'm too tired today. I'll go to the shops tomorrow" or "Tomorrow I'll go to the shops"



\subsection*{Punctuation}
It's standard in the UK to use quote-marks - either single or double ones. There are some conventions -
\begin{itemize}
\item Use a new paragraph for each new speaker
\item The final full stop of a quote is replaced by a comma if there's more text. E.g. -

'I do like you,' he said

\item If you begin with a speech tag, put a comma before the quote. E.g. -The hare said, "I will challenge the tortoise to a race!" (some people use a ":" instead of a comma here)
\item When you have multiple quoted paragraphs, each new paragraph starts with an opening quotation mark, but only the final quoted paragraph has a closing quotation mark.
\end{itemize}

But authors break these rules, and abroad they sometimes do things differently


\begin{itemize}
\item Some authors (e.g. Malcolm Bradbury in "The History Man") don't bother starting new paragraphs for new speakers.
\item Some authors follow the rules above, missing out the quote-marks
\item Some authors (e.g. David Rose) follow the rules above, missing out the quote-marks but adding an initial dash
\item The French and Italians use guillemets - $<< >>$
\item Some languages use this type of punctuation - „May Christ bless this house”
\item Sometimes authors use the method of play scripts
\end{itemize}
Authors aren't even self-consistent from one story to the next. In Anthony Doerr's short story collections, various styles are used -


\begin{itemize}
\item 
\begin{narrow}{1.0cm}{1.0cm}
"Pop," Josh groaned, "those boys are mentally handicapped. I do \textit{not} think some sea-snail is going to cure them."
\end{narrow}
(from "The Shell Collector")


\item 
\begin{narrow}{1.0cm}{1.0cm}
You know her? the hunter asked. Oh no, Marpes said, and shook his head. No I don't. He spread his legs and swiveled his hips as if stretching before a foot race. But I've read her
\end{narrow}
(from "The Hunter's Wife")


\item 
\begin{narrow}{1.0cm}{1.0cm}
She cocks her head slightly. \textit{Look at you. All grown up.}
\textit{I got tickets,} he says.
\textit{How's Mr Weems?}
\end{narrow}
(from "The Deep")
\end{itemize}
Some authors omit quote-marks and some other punctuation characters too. This is from "In a strange room" by Damon Galgut

\begin{narrow}{1.0cm}{1.0cm}
Where have you come from

Mycenae. He points back over his shoulder. And you.
\end{narrow}
Or what about this, the start of "Another country" by David Constantine?

\begin{narrow}{1.0cm}{1.0cm}
 \textit{When Mrs Mercer came in she found her husband looking poorly. What's the 
matter now? she asked, putting down her bags. It startled him. Can't leave 
you for a minute, she said. They've found her, he said. Found who? That 
girl. What girl? That girl I told you about. What girl's that? Katya. Katya? 
said Mrs Mercer beginning to side away the breakfast things. I don't remember 
any Katya.}
\end{narrow}
Or this, from "The Lesser Bohemians" by Eimear McBride (p.120)

\begin{narrow}{1.0cm}{1.0cm}
 \textit{You'll manage all the adulation, he says. Yes, I expect I will. Both go 
Anyway, then laugh and she But what brings you up to these wilds? When steps 
he to show me No! she says}
\end{narrow}

\subsection*{Dangers}
\begin{itemize}

\item \textit{You know, Bob} - 
This is dialogue between characters who share information that they already know, just so readers can get caught up.  Characters don’t have any reason to stand around talking about events they both know about. It's a ploy often used by SF writers to infodump. You're reading an SF novel. After an exciting first chapter set in the 23rd century, there's a scene at a breakfast table. The kids tease Gran about the good old days. She responds by telling them yet again about how tough it was back then, giving a history lesson. But why? The kids have heard it all before.

\item \textit{Monologing, Speeches, Ventriloquising} - at the end of a whodunnit there's often a speech. In other situations though a character launches into a speech that's really what the author should say

\item \includegraphics[width=5cm]{istanbulheads.jpg} \textit{Talking Heads} - All talk, no action.

\item \textit{Ping-pong} - lots of short phrases
\item \textit{Lack of Variety} - The characters shouldn't all speak like you. A radio producer told Emma Darwin  that when he gets a new script the first thing he does is take a ruler, and cover up the left-hand side which shows which character says what. He then reads the play, and if he can't tell who says what without seeing the characters' name, he rejects the play. 

\item \textit{Replacing prose} - In radio drama, dialog is used to describe the scene and action. It's also used to name the characters. If you try too hard to do this on the page, it can seem awkward - you might get away with “Gosh, how long have I been standing in this railway station now?” (From \url{http://www.thewritersguide.co.uk/radio.html} (The Writer's guide)) on the radio, but not on the page. The following is best replaced by description -
"So you’ve decided to fight me, Albert!", "Yes John, and I’m winning, too. I have my foot on your windpipe"

 \end{itemize}



\subsection*{Tips}
\begin{itemize}
\item The commonest advice is Read it out!
\item Watch (and listen to) Drama.


\item "Your characters shouldn't be saying exactly what they're thinking or you give the actors nothing to play."
Marcy Kahan (from \url{http://www.bbc.co.uk/worldservice/arts/features/howtowrite/radio_dev.shtml} (World Service))

\item "Try to remember that as far as possible, characters shouldn't actually answer each other's lines, they should jump off from each other's lines onto something else, or turn corners or surprise people. This will also create movement."
Mike Walker (from \url{http://www.bbc.co.uk/worldservice/arts/features/howtowrite/radio_dev.shtml} (World Service))

\item What's not said is also important.  Silence is more effective on the stage than the page. In prose one may need to use avoidance instead
\item Use dialogue to show deviations from expected conversational norms.
\end{itemize}


\subsection*{Summary and suggestions}
\begin{itemize}
\item Go back to basics. Think about what dialogue reveals about people - not just the words they say, but the pauses, hesitations and interruptions.

\item Read about the recent developments in discourse/conversation analysis. They help make explicit the mechanisms of dialogue we all use.

\item Mainstream literary dialogue has become rather formulaic and artificial. The standard notation hinders the rendering of some revealing aspects of dialogue.

\item Non-standard notations are increasingly common in novels. You might for example consider using screenplay notation.


\end{itemize}



\subsection*{Literary Examples}
\begin{itemize}
\item 
\begin{narrow}{1.0cm}{1.0cm} 
"Does Jack like porridge?" 
"All Scots like porridge!"
\end{narrow}


\item 
\begin{narrow}{1.0cm}{1.0cm}
 “Bring a bottle of wine and wear something uncomplicated – I’m in no mood for 
a struggle tonight,” rolled from Jean-Pierre’s lips like a bowling ball 
shooting up the return ramp, only to slow itself abruptly at the top before 
ka-whonking! into the balls already lined up there like all the lines she had 
heard before, and Sylvia knew at last that all the good ones were not married, 
gay, or in Mexican prisons.
\end{narrow}
James Pokines (the beginning of a novel)

\item 
\begin{narrow}{1.0cm}{1.0cm}
‘Where’s Papa going with that axe?’ said Fern to her mother as they were 
setting the table for breakfast.
\end{narrow}
EB White (the beginning of "Charlotte's Web") 



\item 
\begin{narrow}{1.0cm}{1.0cm}"You're not going out with him and that's the end of it!" Jenny's
father announced.

...

Mrs Wilson winked at her daughter and said: "So he's not such a bad catch
 after all!"
\end{narrow}
The start and end of "It's only rock'n'roll" (Yours, issue 062).

\item 
\begin{narrow}{1.0cm}{1.0cm}
'Very well,' conceded Williamson reluctantly. 'But you are paying.'
'I cannot be long,' warned Chaloner, supposing there was no harm in listening. 
He might learn something useful with no obligation to reciprocate. 'I have an 
audience with the Queen.'
'And you say you have no connections,' said Lester wonderingly.
\end{narrow}
"The Piccadilly Plot", Susanna Gregory, p.280.




\item 
\begin{narrow}{1.0cm}{1.0cm}
"I always liked geography. My last teacher in that subject was Professor 
August A. He was a man with black eyes. I also like black eyes. There are also 
blue eyes and grey eyes and other sorts, too. I have heard it said that snakes 
have green eyes. All people have eyes."
\end{narrow}

In 1911, Bleuler (who coined the term schitzophrenia) quoted this passage from a medical report

\item 
\begin{narrow}{1.0cm}{1.0cm}
The man speaks:
“Should we have another drink?”
“All right.”
The warm wind blew the bead curtain against the table.
“The beer’s nice and cool,” the man said.
“It’s lovely,” the girl said.
“It’s really an awfully simple operation, Jig,” the man said. “It’s not really an operation at all.”
The girl looked at the ground the table legs rested on.
“I know you wouldn’t mind it, Jig. It’s really not anything. It’s just to let the air in.”
The girl did not say anything.
\end{narrow}
“Hills Like White Elephants.”, Ernest Hemingway.

 "In this story, the man is trying to convince the girl to have an abortion (a word that does not appear anywhere in the text). Her silence is reaction enough". (Writer Digest)
\item 
\begin{narrow}{1.0cm}{1.0cm}
'Why?' asks Marty.
Before Lizzie can answer, Robert interrupts sulkily, 'Daddy sent her away.'
'Oh Robert! Don't tell lies!' says his sister, shocked.
\end{narrow}
("The Spoiling", James Lasdun)


\item 
\begin{narrow}{1.0cm}{1.0cm}
'You're far too young for this job. Who sent you to me?'
'Mr Peacock -'
'Dear God, preserve me from do-gooders. Well, boy, do you think you can handle the job? It means a lot of heavy lifting, and you look as though a strong wind would blow you away.'
'I'm a bloody sight stronger than I look - Sir.'
\end{narrow}
("Writing Dialogue - The Essential Guide", Jean Saunders, p.97)


\item 
\begin{narrow}{1.0cm}{1.0cm}
What kind of animals?

He'd sheep. A few cattle, I suppose. Though they'd have been wind-bothered up that way.

They'd have been ...

Bothered, John. By wind coming in. The way it would unseat cattle.

Unseat them?

Cornelius lowers his sad eyes -

In the mind.

You mean you'd have a cow'd take a turn?

Cornelius squares his jaw.

Do you realise you're looking at a man who's seen a cow step in front of a moving vehicle?
\end{narrow}

("Beatlebone" by Kevin Barry)
\end{itemize}

\subsection*{References}
\begin{itemize}
\item \url{http://www.writersworkshop.co.uk/Dialogue.html} (The writers workshop)
\item \url{http://www.writersdigest.com/online-editor/the-7-tools-of-dialogue} (Writers digest)
\item \url{http://howtowriteworkshops.com/writing-dialogue-for-dummies/} (Peter Newland's how to write workshops)
\item \url{http://www.dianagabaldon.com/other-projects/the-cannibals-art-how-writing-really-works/the-cannibals-art-dialogue-workshop-outline/} (Diana Gabaldon)
\item \url{http://www.katherinecowley.com/blog/10-keys-to-writing-dialogue-in-fiction/} (Katherine Cowley)
\item \url{http://hollylisle.com/dialogue-workshop/} (Holly Lisle)
\item \url{http://creative-writing-course.thecraftywriter.com/writing-dialogue/} (The crafty writer)
\item \url{http://www.writerswrite.com/screenwriting/exercises.htm} (Writers write)
\item \url{http://emmadarwin.typepad.com/thisitchofwriting/2011/05/talking-speech-tags.html} (speech tags) (Emma Darwin)
\item \url{http://emmadarwin.typepad.com/thisitchofwriting/2013/03/ping-pong-dialogue.html} (Ping-pong dialogue) (Emma Darwin)
\item \url{http://emmadarwin.typepad.com/thisitchofwriting/2016/12/they-say-my-dialogue-is-weak-what-do-i-do.html} (They say my dialogue is weak. What do I do?) (Emma Darwin)

\item \url{http://www.glimmertrain.com/bulletins/essays/b95macdonald.php} (On Writing Dialogue) (Rowena Macdonald)
\end{itemize}

\subsection*{Sources and Additional Resources on Writing Dialogue}
\begin{itemize}
\item "Writing Dialogue", Tom Chiarella, (Story Press, 1998)
\item "The Write It Write Series: Dialogue Dynamics",  Pinkston, Tristi  (Kindle Ebook, 2012)
\item "Writing Dialogue - The Essential Guide", Jean Saunders  (Need2Know, 2011)
\end{itemize}

\subsection*{Exercises}
\begin{enumerate}

\item Convert the following  (Start of ``Love You Dead'' by Peter James) into dialogue.

The two lovers peered out of the hotel bedroom window, smiling with glee, but each for a very different reason.

The heavy snowfall that had been forecast for almost a week had finally arrived 
overnight, and fat thick flakes of the white stuff were still tumbling down this
 morning. A few cars, chains clanking, slithered up the narrow mountain road, and others, parked outside the hotels, were now large white mounds.


\item Convert the following  (Start of ``Civil and Strange'' by Clair Ni Aonglusa) into dialogue

After Monday morning mass, Beatrice Furlong makes her way toward O'Flaherty's shop. She was never a one for weekday Masses, but now she's like a machine with an
 unforgiving program. Routine and regulation are what save her and sustain her through each day. (\textit{perhaps 2 people could be watching from across the road})

\item Convert this into reported speech (``$>$'' and  ``$<$'' mean ``speeds up''
 and ``slows down''.  ``(.)'' means a short pause)

D: I was (.) at the end of my tether (.) I was (.) desperate (.) $>$I think 
I was so fed up with being $<$ (.) seen as someone who was a ba:sket case (.) 
because I am a very strong person (.) and I \textbf{know} that causes 
complications (.) in the system (.) that I live in. (1s pause) 
((smiles and purses lips))

B: How would a book change that

D: I dunno ((raises eyebrows, looks away)) Maybe people have a better 
understanding (.) maybe there's a lot of women out there who \textbf{suffer} (.)
 
on the same level but in a different environment (.) who are unable to (.)
 stand  up for themselves (.) because (.) their self esteem is (.) cut in two. 
I dunno ((shakes head))

\item Translate the following into a passage that is mostly (or all) dialogue

His arthritis pained him as soon as he woke up, which was maybe why he was so grumpy to his wife over breakfast. By the time he tried to make up it was too late


\end{enumerate}



\newpage
\section{Time and Narration}

A 10 minute story rarely covers 10 minutes of events from beginning to end - some parts are compressed and others expanded. Not only that, but flashbacks and other effects are used to jump backwards and forwards in time. J.M. Coetzee wrote that "For the reader, the experience of time bunching and becoming dense at points of significant action in the story, or thinning out and skipping or glancing through nonsignificant periods of clock time or calendar time, can be exhilarating - in fact it may be at the heart of narrative pleasure''. I think some 
short-story writers underuse these effects, so I'd like to talk about them now. 
 


\subsection*{Speed}
Changes of speed are so common in all forms of storytelling that we hardly notice them. Here are some examples



\begin{itemize}
\item  compression: "So we lived in Texas for five years, and then we moved to California."
\item  expansion: "All of the sudden it occurred to me in a flash of insight that she never really loved me and had only been using me to make her husband jealous and to ensure that one way or another she could get her green card. How could I have been so stupid, how could I have courted such a disaster?"
 \end{itemize}
Passages of dialogue bring us back to real time. Thriller writer Lee Child said "write the fast slow and the slow fast" (i.e. write the fast-action scenes in slow motion and gloss over the long, boring journeys, etc)



 
\subsection*{Direction}
Our thoughts are rarely satisfied to stay settled in the present moment; instead, they tend to wander nostalgically into replays of past scenes, or to fantasize about the future. So it's natural that authors go back and forwards in time. The flashback [analepsis] is quite common. Flashbacks


\begin{itemize}
\item  help give short stories the illusion of depth
\item  help to "show, not tell"  - rather than mention that someone used to be
a soldier, flash them back to a battlefield
\item  can be used at the start of a story to capture the reader's interest.
\end{itemize}
but they have disadvantages too
\begin{itemize}
\item  they interrupt the momentum of the story
\item  overused, they can disorganise the story, especially if there's
no present to contrast them with. It helps to use them right at the start or to fully establish the characters first
\item  the choice of tense to use can be tricky. Authors usually begin a flashback in the past (or pluperfect) tense then drop it once the flashback is established
\end{itemize} 
Flashbacks are typically provoked by


\begin{itemize}
\item  going through an old photo-album or diary - see "Krapp's Last Tape" (Beckett)
\item  finding an object you haven't seen for years
\item  revisiting a place where you used to live
\item  a taste or smell - Proust's madelaine
\end{itemize}
and ended by an interruption from the present. Flashbacks can be extensive. Sometimes the first chapter of a novel is a flashback, but you don't find that out until later. Sometimes most of a story is a flashback framed by the words of the narrator or author. Sometimes the flashbacks and the present alternate through the piece.



 
A special case of the flashback is the story-within-a-story [or intercalated story]. Detective stories use this idea quite a lot - each witness giving their version of the events.



 
Less common than flashbacks are glimpses into the future. These might seem to spoil the surprise, but often it increases anticipation



\begin{itemize}
\item  foreshadowing [or premonition, prefiguration]: short hints about the future - "grey shadows portending deeper shadows to come.", "little did they know, as they kissed on the platform, that they'd never meet again". These are sometimes used at the ends of sections to encourage the reader to continue. Sometimes however, it takes a second reading to discover them.
\item  flashforward [or prolepsis]: "Many years later, as he faced the firing squad, Colonel Aureliano Buendia was to remember that distant afternoon when his father took him to discover ice." (opening line of Garcia Marquez' One Hundred Years of Solitude)
\item  adumbration - in older works, chapters often had titles or summaries. For example Galsworthy uses chapter titles like "Soames Breaks the News".
\end{itemize} 
Finally there's repetition - a word, gesture or memory used as a leitmotif having the effect of making time cyclical.
 


\subsection*{The Short Story}
Opinions differ on whether flashbacks work in short stories


\begin{itemize}
\item  "Flashback is almost always necessary at some stage in the writing of a short story" - "Practical Short Story Writing", John Paxton Sheriff, p.83
\item  "In writing a short story, the flashback should probably not be used", "Guide to Fiction Writing", Phyllis Whitney, p.113
\item  "Confession stories nearly always need a flashback", "How to Write Stories for Magazines", Donna Baker, p.45
\end{itemize}
 
It's easier to use direction-changing in novels where there's more room to explain what's going on and chapters provide handy dividing lines. In the short story rapid jumps might confuse the reader. On the page, italics and roman text could be used to show the transitions, but it's not common. Breaking the story into short sections with subtitles can help too. 



 
One tip from Sol Stein ("Stein on Writing", p.144) is that the first sentence of a flashback needs to be arresting to jolt the reader from what went before.



Foreshadowing is sometimes added (especially in later drafts) to give the work more unity (see the Old Testament rewrites, for example). In "The Great Gatsby" the foreshadowing is unlikely to be noticed on a first reading but they add to the sense of inevitability.


 
 
\subsection*{Examples}
I've already quoted a few examples. Here are some more


\begin{itemize}
\item  "Time's Arrow" - Martin Amis. In this book time goes backwards. Food is
taken out of the mouth, put on the plate and eventually taken to the shops in
return for money. In Vonnegut's "Slaughterhouse Five", Dick's "Counterclock
  World" and Alexander
    Masters' "Stuart: A life backwards" the device is used to a lesser extent. Courttia Newland's short story "Reversable"  from Faber's "Sex and Death" anthology (2016) is told in reverse.
\item Steven Maxwell's short story "The Fade" in "Staple" (issue 73) begins "At
  seven in the morning, as the sun was setting, his wife's expansions
  began". Later in the Departure Room something is pushed into the wife - "'The
  placenta', said the midwife, ... 'Just making the bed, so to speak.'". They
  go home, dashing through red lights. The story ends with "But for now they
  are content just to be doing their best for the baby, whoever it was, and
  making its fade as painless as possible. And in nine months time, when his
  wife has ejaculated her seed into him, all will be forgotten, the fade will
  be complete."
\item "Otto Grows Down" by Michael Sussman is a children's picture book where the child, Otto, experiences time in reverse after his baby sister is born 
\item "First Light" by Charles Baxter goes from the main characters' middle-age to their childhood.
\item "The Night Watch" by Sarah Waters starts in 1947, then goes to 1944 and 1941.
\item  "A Rational Man" (Teresa Benison) uses various tricks.
\item  "Wuthering Heights" (Emily Bronte) uses flashback.
\item  "Beloved" (Toni Morrison) uses flashback.
\item  "Nostromo" (Conrad) and "The Prime of Miss Jean Brodie" (Spark) uses flashforward.
\item  "Turn of the Screw" (Henry James) is a framed story.
\item  "The Sound and the Fury" uses various narrators describing the same events.
\item  "The Time Traveler's Wife" has a man who travels backwards and forwards in time. The reader's given chronological information - a sample section heading is \textit{Friday, June 5, 1987 (Clare is 16, Henry is 32)}
\item Iain Banks' "Use of Weapons" has 2 narrative threads - one going forwards in time, one going backwards.
\item The sections of de Lillo's "Underworld"  are in reverse order
\end{itemize}
Next time you read a story, look out for the changes in narrative speed and direction. It's quite common for narration speed to match chronological speed at the climax of the story. 


 
Also look at how films use the same tricks. Several films I've recently seen
("Saving Private Ryan", "Cinema Paradiso", "La Vita e Bella", "Eternal Sunshine
Of The Spotless Mind") use flashbacks extensively. "Memento" intersplices 2
story-lines, one going backwards and the other forwards. Directors can switch between colour and monochrome to show the transitions. Compression is harder though, requiring voice-overs or a caption saying something like "5 years later".



\subsection*{See Also}
\begin{itemize}
\item "The Art of Time in Fiction", Joan Silber, Graywolf
\end{itemize}
\newpage
\section{People who need people (character workshop) }
In some types of prose, the author needs to know their characters well, so well
that the author might recognise them if they saw them in the street. Today
we'll try some workshop exercises on character development. We won't have time
to produce any finished work, but maybe one or two of the characters created
today will walk into your next story. We'll first look at a simple way to
create viable characters, then look at character development in more detail.



\subsection*{First impressions, habits and surprises}
\subsubsection*{Exercise 1 - Hi!}

When new characters appear in some novels they're immediately described. Here are a few shortened examples from "An Unsuitable Job for a Woman" by P.D. James


\begin{itemize}
\item  Leaming - "wearing a grey suit with a small stand-away collar which showed a narrow band of white cotton at the throat ... She was tall and her hair, prematurely white, was cut short and moulded to her head like a cap. Her face was pale and long" (p.17)
\item  Lunn - "stockily built young man dressed in an open-necked white shirt, dark breeches and tall boots ... large mud-brown eyes ... beautiful, moist calves' eyes heavily lashed and with the same look of troubled pain at the unpredictability of the world. But their beauty emphasized rather than redeemed the unattractiveness of the rest of him" (p.22)
\item  Marklands - "All three reminded her of horses. They had long, bony faces, narrow mouths about strong, square chins, eyes set unattractively close, and grey, coarse-looking hair" (p.40)
\item  the Tillings - "strong dark heads held high on usually short necks, and their straight noses above curved, foreshortened upper lips" (p.72)
\item  de Lasterie - "an oval face with a neat slender nose, a small but beautifully formed mouth, and slanted eyes of a striking deep blue which gave her whole face an oriental appearance at variance with the fairness of her skin and her long blonde hair." (p.72)
\item  Stevens - "a stocky, bearded young man with russet curly hair and a spade-shaped face" (p.72)
\item  police surgeon - "a fat, dishevelled little man, his face crumpled and petulant as a child when forcibly woken from sleep" (p.182)
\item  Dalgliesh - "tall, austere ... over forty at least ... dark, very tall and loose-limbed ... His face was sensitive without being weak" (p.208)
\end{itemize}

How would P.D.James describe you?



\textit{Get people to do this on sheets of paper. Collect in results. Read a few
    out and see if people can guess who the person is.}



 Are all the senses used? Is it a good literary style? It's realistic in the sense that when you meet someone you might first recognise a face, a name, and a voice. But what then?
How can you make your characters believable so that readers care about them?
One answer is "make them like real people". But what are real people like? I 
decided to find out by going to the library.
In the Cambridge central library there's a book called "Creating Fictional
  Characters" by Jean Saunders. Who is this Jean Saunders person? When I read
  that she's "\textit{written well over 100 novels ... She is also a frequent
    enthusiastic lecturer on cruise ships}" my heart sunk, but actually the
  book's ok. She points  out that for characters to be convincing they may need
  to be consistent (for example, their name shouldn't change during the story!). But she also has a section called "Unstereotyping the stereotype". People are a mix of the predictable and the surprising. The predictable features aid identification and empathy, making surprises possible. E.M. Forster wrote that "the test of a round character is whether it is capable of surprising in a convincing way."
The next 2 exercises will focus on these issues




\subsubsection*{Exercise 2 - Habits aren't boring}
P.D. James focused on faces - long, oval, spade-shaped, etc. Many of us do,
sometimes at the expense of other features. A colleague at work has slight
face-blindness. He says he's "very good at recognising people from a distance
(from their posture / how they walk / etc)". They say that the blind have a
more acute sense of hearing. He's probably got a more acute sense of
mannerism. Writers need to develop a similarly acute sense. People's mannerisms
help define them and can be used to develop a leit-motif - the way they keep
fluffing their hair up, how they play with a pen. What to do they do with their
other hand while brushing their teeth? When they talk on the phone do they
doodle? What to they do while waiting in a supermarket queue? Who would never
put their hands on their hips? Who often does it?  These habits are often hard
to notice and shouldn't be underestimated. You may only realise them when
they're broken - an early signs of mental problems perhaps, those little things
that can be so revealing to those who know.

\begin{narrow}{1.0cm}{1.0cm} 
Write down some gestural habits of people you know (you might be so used to 
them that you don't notice them?) or have seen on TV. What features might 
impressionists pick up on? They might be things that others do as well, or 
they might be unique. I'm interested with what sort of actions you come up with.
\end{narrow}


\subsubsection*{Exercise 3 - Quirks, Secrets and Surprises}
You can try to make characters "interesting" by making them eccentric, but little quirks and secrets are more useful. 
A character's guilty secrets can bring a character to life. If you know a character well, you'll know these things about them.


\begin{narrow}{1.0cm}{1.0cm}
What's in the zipped compartment of a character's handbag or wallet? What TV 
programs do they guiltily stay in for? What do they eat when alone that they 
don't eat with others? Which member of their family have they actually never 
really liked? What are your secrets?  How do you give yourself a treat? List 
the answers (or those of a "friend"). Alternatively, write a paragraph or 2 
(or a plot idea) about what happens when a person's secret is suddenly discovered.
\end{narrow}

\subsection*{Character construction and development}
We'll take a break from workshopping for a moment to look at character development. What is a "memorable character"? When someone has "bags of personality" what does that mean? Jimmy Savile was described as a "larger than life character" - why? Who has charisma? Mandela? Hitler?


Here's one way to build characters up


\begin{itemize}
\item  Create a scrapstore of bodyparts and behaviours - habits, quirks and facial features. This is something that actors do.
Where can you go to pick up ideas? 
\begin{itemize}
\item  Magazines - "Hello!" magazine
\item  TV - Reality TV; fly-on-the-wall TV; The Fast Show; YouTube. Forget about plot. It might help to turn the sound off.
\item  People-watching - remember, it's parts of behaviour you're interested in, not complete people.  
\item  Lifestyle consultants - they're the experts. Books like Edward de Bono's "How to be more interesting" offer tips 
\end{itemize}
Us writers are lucky - if we're attending a tedious business meeting or social event we can start people-watching, collecting material.
If your spouse accuses you of ogling, reply that you're just doing research. If a workmate wonders why you're reading, "How to be more interesting" say it's not for you, it's for a character in your novel.

\item  Construct a character from the bits and behaviours. Avoid the temptation to make everything symbolic, to make everything fit neatly together. Barthes suggested that by adding details that are non-literary and arbitrary, with no symbolic value you can create a "ring of truth" - "The Reality Effect".

\item  The result might initially be a Frankenstein's Monster. Don't worry. Wait for a bolt of lightning to bring the character to life. Once the character moves, the useless bits will drop off as you re-write
\end{itemize}

You might feel that the resulting character's contrived but it works well
enough for me! Some rough edges don't do any harm - they add to the realism.
In my poetry book I've got deathbed scenes, etc. One reviewer said my stuff had the "unmistakeable authority of experience"; "The strength of the personae in the pamphlet is the thing that attracts attention" which is embarrassing because my stuff's all lies. 




A character needs something that sparks them into life. An experience can be character-building, and they say "cometh the hour, cometh the man". The next bunch of exercises looks at situations, language and settings that can give birth to characters or enhance them. Remember, any advice you might give to a friend might also be applied to your characters, and vice versa.





\subsubsection*{Exercise 4 - Show not tell}
Where do characters come from? Sometimes, especially in a first novel, they come from inside the author. Such characters can end up looking and sounding all the same. Characters are often copied from relatives or friends. This can be risky, even if traits are combined and genders changed (it's been claimed that "The Godfather" was based on the author's mother). Let's try an exercise on that theme
\begin{narrow}{1.0cm}{1.0cm}
Think about your mother or father. Make a list of 3 qualities that describe him
or her. Write a paragraph that captures some or all of those characteristics
through significant detail. \textit{No word on your list should appear in the
paragraph}
\end{narrow}

\textit{Get people to do this on sheets of paper and swap papers afterwards. Get
    people to guess the qualities.}




\subsubsection*{Exercise 5 - Tales of the Unexpected}
\begin{narrow}{1.0cm}{1.0cm}

You can find out a lot about people by how they react to unusual situations.
Write a paragraph (a scenario plus plot) about a family member meeting a famous person unexpectedly.

Alternatively, at customs they find something strange (not illegal or rude) in your character's luggage.
Your character has to explain how it got there. Invent a scenario.

\end{narrow}


\subsubsection*{Exercise 6 - On the spot}
Once you've created your characters you can develop them by interviewing them as if they were real. You can find some questionnaires online. Gotham Writers' Workshop offers a questionnaire with questions like 


\begin{itemize} 
\item  What makes your character laugh out loud?  
\item  What is her biggest fear? Who has she told this to? Who would she never tell this to? Why? 
\item  Your character is doing intense spring cleaning. What is easy for her to throw out? What is difficult for her to part with? Why? 
\end{itemize}
\begin{narrow}{1.0cm}{1.0cm}
Suggest some questions that might provoke interesting responses.
\end{narrow}

\subsubsection*{Exercise 7 - What's in a name?}
JK Rowling has revived interest in character names. Non-fantasy writers also need to be careful when naming characters. Invent a few names, or write your first reaction to one or more of these names. Which ones would you hesitate to go on a blind date with? Lavinia Blackmun,  Wladziu Valentino, Sharon Smith,  John Thomas, Florian Cloud de Bounevialle Armstrong. 


\begin{narrow}{1.0cm}{1.0cm}
Think up a character type and a genre. Now think up a name. 
\end{narrow}

(\textit{Florian Cloud de Bounevialle Armstrong is Dido; Wladziu Lee Valentino was Liberace}).




\subsubsection*{Exercise 8 - Scene, then heard}
If you create sufficiently vivid locations, characters will emerge


\begin{narrow}{1.0cm}{1.0cm}

Close your eyes.
Try to imagine yourself in the kitchen where you grew up as it was then. Look 
around. How many chairs are there? Where's the tea-pot kept? (on the fridge 
on a knitted tea cosy). Where's the litter bin (by the back-door; it's a 
pedal-bin with a broken pedal). What's on the top shelves at the back? (spare keys).


Now have your mother walking in as she was then. She thinks you're the home
 help. She asks what you're looking for. What would you say?
\end{narrow}

\subsubsection*{Conclusions}
Once you've got a character, what next?
Jean Saunders writes - "I sincerely believe that you must be prepared to love all the characters
you create" but I think that's going a bit far. Alternatively you can


\begin{itemize} 
\item  treat them "like galley slaves" (Nabokov) 
\item  put them in a room and wait to see what happens (Beryl Bainbridge).
\item  Fitzgerald, somewhere between the 2, said that "Character is plot, plot is character".
\end{itemize}

One tip though - don't do all the character development beforehand; leave some to happen during the story. The characters at the end shouldn't be the same as the characters at the start.



I hope you too have been changed a little by this evening.



\subsection*{See also}
\begin{itemize}
\item \url{http://emmadarwin.typepad.com/thisitchofwriting/2013/03/characterisation-in-action.html} (Characterisation-in-action) (Emma Darwin) 
\end{itemize}
\newpage
\section{From Words to Flash}

This evening I hope to illustrate that there's more to sentences than might first appear. Like stories, they can have tension and resolution, pacing and surprise. Getting a sentence right can take a long time but it's worth it - if an editor sees a bad sentence in the first paragraph, your script is likely to go straight in the SAE. We're going to look at words first, then try to calibrate our sensibilities by looking at good and bad sentences, then we'll try to assess some sentences before trying to write our own. At the end we might get as far as writing little stories.


As we'll see later, people disagree about the goodness of sentences. I'm going to focus more on form than content; I'm not going to highlight sentences for their wisdom.



\subsection*{Words}
However complex the sentences, they're always made up of words, so we'll start by sensitizing ourselves to the building blocks though we're not going to focus on individual words today.





\begin{narrow}{1.0cm}{1.0cm}
\textbf{Exercise 1} What are your favorite words (you may like them for their meaning or merely their sound). What are your least favorite words.
\end{narrow}

\textit{Dylan Thomas liked "drome". Some people don't like "garaged" or "to medal"}




In poetry books you'll sometime see provocative lists of forbidden words ("shards", "gossamer", "fester", "frond", "lambent","shimmer" etc) but of course, it's not that easy. They say of weeds that they're only flowers in the wrong place. One might say a similar thing about unpopular words. 




One way to try to improve a sentence is to replace some of the words by more interesting ones mined from a thesaurus - replacing "red" by "vermillion", "road" by "thoroughfare", "bird" by "lesser spotted warbler" etc. Alas, a sentence full of pretty words might not be pretty. One misplaced word, however interesting, might destroy a sentence. 




Today I'm going to work on the assumption that "if you look after the sentences the words will look after themselves". If you want to see how tackle a sentence word by word, take a look at 
 \url{http://jim-murdoch.blogspot.com/2010/06/how-to-write-sentence.html} (Jim Murdoch's blog) where  Jim Murdoch tells us how he spent 3.5 hours trying to get a sentence right, or look at 
the University of Rouen's collection of \url{http://www.bovary.fr/folio_visu.php?mode=sequence&folio=&org=3&zoom=50&seq=91} (Flaubert's scripts)




\subsection*{Sentences}
Now we're going to study some sentences. First we'll look at the extremes


\subsubsection*{Bad sentences?}
In "PN Review 185" Frederic Raphael spends 4 large pages attacking the prose quality of Ian McEwan's "On Chesil Beach". For example, he quotes his sentence -"Two youths in dinner jackets served them from a trolley parked outside in the corridor, and their comings and goings through what was generally known as the honeymoon suite make the waxed oak boards squeak comically against the silence", commenting "'Parked outside in the corridor' is another stock phrase, in which 'outside' is superfluous and 'parked' witless. The repeated use of the copular after the comma appends a  tail too long to be supported, most of it ('comings and goings … generally known as the honeymoon suite … comically') redundant. Who cares what the honeymoon suite was generally called when we are already in it? If you use your imagination, you may guess that it had a considerably less brochuresque nickname among the staff, whose consciousness we are supposed to be sharing. Let it pass; what was comic about the squeak, if comic is right, has surely to be that someone - Florence, Edward, the 'youths' - found it so. In which case, in whom and how did the comedy express itself?" (p.45).



I've seen far worse sentences than that. Some of the ones below come from famous writers -



\begin{narrow}{1.0cm}{1.0cm}
\textbf{Exercise 2} - Can you improve these sentences? Do any make you wince?
\end{narrow}

\begin{enumerate}
\item Having pitched the tents, the horses were fed and watered
\item Before describing what happened, the background to these events must be understood
\item  What a beautiful sky, she thought to herself as she ran slowly along the narrow road.

\item  More than one person lives in this house.

\item  On holiday her car got scratched.

\item  On a late winter evening in 1983, while driving through fog along the Maine coast, recollections of old campfires began to drift into the March mist, and I thought of the Abnaki Indians of the Algonquin tribe who dwelt near Bangor a thousand years ago. 

\item  She wore a dress the same color as her eyes her father brought her from San Francisco

\item  Craig stared into the mirror and squinted his eyes, for a moment he could almost see his brother staring back at him.

\item  There were less people there than I'd expected.


\item  Abstracting the face of the student from the file, the probationer took it to his superior.

\item  He and his group swelled in number.
\end{enumerate}
\textit{
There's an old joke - "I know a man with a wooden leg called Jim". "What's the name of his other leg?". 1 and 2 are from "A Student's Guide to Writing" (Gordon Taylor). 6 is by Norman Mailer -  first line of "Harlot's Ghost". 7 is from "Star" by Danielle Steel. The last two are from "The Afghan" by Frederick Forsyth
}



\subsubsection*{Good sentences?}

Joseph Epstein chose this as the best first sentence in literature - "Happy families are all alike; every unhappy family is unhappy in its own way." (Anna Karenina, Tolstoy).
Fish's "How to Write a Sentence and How to Read One" (HarperCollins, 2011)  doesn't mention this. He suggests that you use good sentences as models. 


\begin{narrow}{1.0cm}{1.0cm}
\textbf{Exercise 3} - Study 2 of Fish's favorites
\end{narrow}


\begin{itemize}
\item "And I shall go on talking in a low voice while the sea sounds in the
distance and overhead the great black flood of wind polishes the
bright stars." (Joyce)
\item "So we beat on, boats against the current, borne back ceaselessly
    into the past." (Fitzgerald).
\end{itemize}



\textit{Here's one response to the Fitzgerald sentence: I love how it's so tantalizingly close to iambic pentameter - 5 iambs
followed by 4 and 1/2.The cadence carries the reader forward in the
first phrase with four staccato syllables. The choppiness of the
second phrase brings the current's restraint to life, interrupting the
flow of the sentence. The final phrase glides easily, but the missing
twentieth syllable leaves the reader anticipating more. One can
imagine the novel's last sentence repeating endlessly, beginning again
where it left off. And of course that's the point. The art of the
sentence is in its structure as much as its words.}


 
\subsubsection*{How to study sentences}
I think the quality of a sentence isn't as objective as the previous comments imply. Let's evaluate some sentences, learn how to talk about sentences. Punctuation is important, so we'll consider that too.



\begin{narrow}{1.0cm}{1.0cm}
\textbf{Exercise 4} Punctuate and study this - It is a truth universally acknowledged, that a single man in possession of a good fortune, must be in want of a wife.
\end{narrow}

\textit{With this and similar exercises supply it without punctuation and get them to guess the author and add punctuation. This the first sentence of Jane Austen's "Pride and Prejudice"}



This dates from a time when books were often read out. Punctuation indicated where to pause, rather than indicating logical structure.


\begin{narrow}{1.0cm}{1.0cm}
\textbf{Exercise 5}  Punctuate and study this - When they entered the room, he was very dead,
maybe three or four days dead, since no one at the hotel had seen him around for some time
\end{narrow}


\textit{It's by Margot McCamley and it comes from Writing Magazine, May 2012. They have an "Under the Microscope" feature where they study the first 300 words of a novel.}




Here are some suggestions by James McCreet


\begin{itemize}
\item  Start with "He was very dead when they entered the room ..."
\item  "The sentence as it stands cannot use a comma before maybe"
\item  "quite halting and convoluted. This final clause overloads it with information when it needs to have impact and focus"
\end{itemize}

\begin{narrow}{1.0cm}{1.0cm}
\textbf{Exercise 6} Punctuate and study  - 
The April day was soft and bright, and poor Dencombe, happy in the
conceit of reasserted strength, stood in the garden of the hotel,
comparing, with a deliberation in which, however, there was still
something of langour, the attractions of easy strolls
\end{narrow}


\textit{It's the first sentence of Henry James' "The Middle Years"}



Note that there are lots of opposites. Here are some notes about it from "next word, better word" by Stephen Dobyns (Palgrave Macmillan, 2011)


\begin{itemize}
\item     "James used these commas to call attention to important words,
    used them in fact as line breaks are often used in poetry", p.129
\item      "begins with an independent clause; the tone is straightforward
    and somewhat optimistic ... Rhythmically, we notice the clause is
    four iambs, which contributes to its lightness.", p.129 
\item  "The
    second independent clause has seven commas, which ensures no
    consistent rhythm can be established. This rhythmic disruption, as
    it were, arises directly from the word "poor" ... Dencombe
    is "poor" because of his health, but also because he is
    deceived.", p.129 
\item  "The modifying phrase between the subject,
    Dencombe, and the verb, "stood," the following dependent clause
    and string of prepositional phrases create tension by delaying
    verbs and direct objects, but they also in their progression and
    rhythm imitate the languor of Dencombe's thought [which] leads to
    a slightly humorous direct object", p.129 
\item  "As with a classic
    Latinate sentence, James's second independent clause accumulates
    meaning until it reaches its most important words.", p.130
\item   "James's sentence keeps us from being able to anticipate its
    direction and controls the speed at which we read it, while the
    word "poor" provides us with suspense enough to care about that
    direction", p.130 
\end{itemize}

\begin{narrow}{1.0cm}{1.0cm}
\textbf{Exercise 7} Punctuate and study this sentence - There was a little stoop of humility 
as she passed through the door, into the larger but darker library beyond, a 
hint of frailty, an affectation of bearing more than her fifty-nine years, a 
slight bewildered totter among the grandeur that her daughter now had to 
pretend to take for granted.
\end{narrow}

\textit{ From "The Stranger's Child", Alan Hollinghurst }



\textit{Here are some notes - "humility" has rather quickly elided into "affectation," and the point of view has shifted by the end of the sentence, and the physical movement through the rooms accompanies a gradual inner movement that progresses through four parallel clauses, each of which, though legato, suggests a slightly different take on things}




Note how these sentences have a shape - they have tension and release, the form of the sentence emulates the mood. They flow. But try this




\begin{narrow}{1.0cm}{1.0cm}
\textbf{Exercise 8} Punctuate and study this sentence - 
Oddly enough, she was one of the most thoroughgoing sceptics he had ever met, 
and possibly (this was a theory he used to make up to account for her, so 
transparent in some ways, so inscrutable in others), possibly she said to 
herself, As we are a doomed race, chained to a sinking ship (her favourite 
reading as a girl was Huxley and Tyndall, and they were fond of these nautical 
metaphors), as the whole thing is a bad joke, let us, at any rate, do our part;
 mitigate the sufferings of our fellow-prisoners (Huxley again); decorate the 
dungeon with flowers and air-cushions; be as decent as we possibly can.
\end{narrow}
\textit{
Virginia Woolf, Mrs Dalloway (more an exercise in punctuation). If you have the time, try this
}



\begin{narrow}{1.0cm}{1.0cm}
\textbf{Exercise 9} Punctuate and study this sentence - No, I don't know them," he said, but instead of vouchsafing so simple a piece of information, so very unremarkable a reply, in the natural conversational tone which would have been appropriate to it, he enunciated it with special emphasis on each word, leaning forward, nodding his head, with at once the vehemence which a man imparts, in order to be believed, to a highly improbable statement (as though the fact that he did not know the Guermantes could be due only to some strange accident of fortune) and the grandiloquence of a man who, finding himself unable to keep silence about what is to him a painful situation, chooses to proclaim it openly in order to convince his hearers that the confession he is making is one that causes him no embarrassment, is in fact easy, agreeable, spontaneous, that the situation itself--in this case the absence of relations with the Guermantes family--might very well have been not forced on, but actually willed by him, might arise from some family tradition, some moral principle or mystical vow which expressly forbade his seeking their society.
\end{narrow}
\textit{
Proust. Translation by C.K.Scott Moncrieff and Terence Kilmartin.
}



\subsubsection*{Sentence length}
These sentences are growing longer. Is there any limit?



\begin{itemize}
\item  Jeet Thayil’s debut novel Narcopolis (ManBooker-shortlisted in 2012) begins with a 5 page sentence
\item  "The Last Voyage of the Ghost Ship" by Gabriel Garcia Marquez (5 pages)
\item  Molly's soliloquy in "Ulysses"
\item  Jonathan Coe's novel "The Rotters' Club" (2001) ends with a 33-page sentence (13,955 words)
\item  Ed Parks' novel "Personal Days" ends with a sentence over 16,000 words long.

\item  "Dancing Lessons" (Hrabal) is a single 117 page sentence
\item   "The Gates of Paradise" (Jerzy Andrzejewski, 1960) consists of two sentences, the first 158-pages long and the final one 5 words.
\end{itemize}

\subsubsection*{Writing sentences}
We get into habits when writing sentences. The next exercise strives to break those habits by varying structure while keeping the words the same. 



\begin{narrow}{1.0cm}{1.0cm}
\textbf{Exercise 10}  Use the all the following phrases in 1st person, past tense sentences.  No embellishments - just stick to the given information. "Jim", "purple armchair", "blow his tea cool" (or "blowing his tea cool", or "the better to blow his tea cool"), "talking" (or "talked"), "Arabella", " resigned amusement on her face" or ("an expression of resigned amusement on Arabella's face"), "walked into the room" (or "walking into the room"), "I saw". First write the sentence that comes most naturally to you, then try some variations
\end{narrow}
\textit{Adapted from an Emma Darwin blot post. Get people to produce sentences, then discuss the ones below}



Here are some of the alternatives you could have produced. Which is best?



\begin{enumerate}

\item  I walked into the room and saw that Jim had sat down in the purple armchair, the better to blow his tea cool and talk to Arabella, who had an expression of resigned amusement on her face.

\item  Walking into the room I saw that Arabella, with an expression of resigned amusement on her face, was being talked to by Jim, who had sat down in the purple armchair the better to blow his tea cool.

\item  Arabella had an expression of resigned amusement on her face, as I saw when I walked into the room, and was talking to Jim, who was blowing his tea cool as he sat in the purple armchair.

\item  Jim was sitting down in the purple armchair and talking to Arabella, who had an expression of resigned amusement on her face while she watched him blowing his tea cool, as I saw when I walked into the room. 

\item  The resigned amusement on Arabella's face, as I saw when I walked into the room, was caused by watching Jim sitting in the purple armchair blowing his tea cool and talking to her.

\item  Talking to Arabella, who had an expression of resigned amusement on her face as I saw when I walked into the room, and blowing his tea cool, was Jim, who was sitting in the purple armchair.

\item  Sitting in the purple armchair I saw Jim, who was blowing his tea cool, and the resigned amusement on Arabella's face as he talked to her which I saw when I walked into the room.

\item  Sitting in the purple armchair and talking to Arabella was Jim, blowing his tea cool, and I saw when I walked into the room that she had an expression of resigned amusement on her face.

\item  Blowing his tea cool as he sat in the purple armchair was Jim, and I saw as I walked into the room that he was talking to Arabella, who had an expression of resigned amusement on her face.
\end{enumerate}

Emma Darwin suggests that \textit{A good basic principle is to stick with the order in which your view-point-character perceives things, which is likely to be 1) in this example. 4) is an example of what I call 'zig-zagging', which starts a little way in and then jumps back to the beginning.} Would your narrator take more notice of Jim or Arabella? Would s/he observe their emotion along with their action, as in 5), or only after a general survey of the room, as in 1)? 7) is a mess.




If you were allowed to use any words in this sentence, how might you write it? Perhaps you could hint that there's something about the situation that surprizes the narrator.




\subsection*{Short Fiction}

If you go through each sentence in a story of yours, generating a list of alternatives and assessing them, it might take a while. Maybe it's time to consider shorter forms.  There are many to choose from - memoes, recipes, adverts, anecdotes, lists, vignettes, shopping lists, etc. More recently there's "ketai fiction" (to fit in a text message), "Twitter Lit" (to fit Twitter's 140 character feed), etc. There used to be markets for these genres (fillers, Readers Digest). In his collection "This Isn't The Sort Of Thing That Happens To Someone Like You" (Bloomsbury, 2012), Jon McGregor has 2 stories that are fewer than 15 words long. Markets are re-emerging. Let's first try an easy genre.



\subsubsection*{6 word stories}
How many words do you need for a story? 6 might be enough.  Examples include


\begin{itemize}

\item  For sale: baby shoes, never worn -  Hemingway

\item  Computer, did we bring batteries? Computer?  - Eileen Gunn

\item  Longed for him. Got him. Shit.  - Margaret Atwood
\end{itemize}

\begin{narrow}{1.0cm}{1.0cm}
\textbf{Exercise 11} Write some 6 word stories
\end{narrow}
\subsubsection*{Flash}

What is Flash? According to the Bridport competition it's a maximum of 250 words that "contains the classic story elements:
protagonist, conflict, obstacles or complications and
resolution. However unlike the case with a traditional short story,
the word length often forces some of these elements to remain
unwritten: hinted at or implied in the written storyline." It overlaps free-form poetry at one extreme, and short stories at the other. Pieces like Forche's "The Colonel" have appeared both in Flash and poetry anthologies. Often the limit's more than 250 words - 1000 is common.



Here are 2 from the web



\begin{itemize}
\item \textbf{Bedtime Story}

"Careful honey, it's loaded," he said re-entering the bedroom.

Her back rested against the headboard. "This for your wife?"

"No. Too chancy. I'm hiring a professional."

"How about me?"

He smirked. "Cute. But who'd be dumb enough to hire a lady hit man?"

She wet her lips, sighting along the barrel. "Your wife." 

(Jeffrey Whitmore)

\item The last man on Earth sat alone in a room. There was a knock on the door... (Fredric Brown)
\end{itemize}

\begin{narrow}{1.0cm}{1.0cm}
\textbf{Exercise 12} Put people in pairs. Get them to think up a scenario where someone has 
to explain an awkward situation. Get them to write an Abcedarian, each person 
writing alternate sentences - begin the first sentence with A, the second with 
B and so on.
\end{narrow}
\textit{Impro comedians can do this on the spot}


\subsection*{Conclusions}
We've looked at individual words and how their energy can be harnessed in a carefully punctuated sentence. The structure conveys its own meaning, like a bass-line mood accompaniment to the melody of the words' meaning. Not all readers will notice this bass-line, but they'll feel it, and some editors will look out for it.




We've looked at what to avoid and emulate, how length and clause order can be varied. What next?  As an exercise, print out a story of yours one sentence per paragraph and consider each sentence in isolation. Are they of similar length? Do they all have a subject-verb-object structure? After labouring so long over so few words, you may no longer have the stamina to write 3000 word stories. Don't worry - shorter stories will do!



\begin{itemize}
\item  When a competition maximum word limit in 3000, don't forget that a 500 word story might win.
\item  You might be able to get your piece accepted as a prose-poem. 
\end{itemize}
Finally - if nothing else, get the key sentences right!




\subsection*{Some URLs}
\begin{itemize}
\item  \url{http://www.sixwordstories.net/} (sixwordstories) and  \url{http://narrativemagazine.com/} (narrative magazine's 6-word stories)
\item  \url{http://www.nytimes.com/2010/12/26/books/review/Park-t.html?_r=1} (about long sentences)
\item  \url{http://www.newcriterion.com/articles.cfm/Heavy-sentences-7053} (about sentences)
\item  \url{http://wordswithoutborders.org/article/el-ultimo-lobo} (a long sentence)
\item  \url{http://onesentence.org/} (one-sentence stories)
\item  \url{http://www2.eng.cam.ac.uk/~tpl/lit/flashoutlets.html} (a list of Flash outlets, sorted by word-length)
\end{itemize}

With help from John Riley, Janice D. Soderling et al.



\newpage
\section{Point of view}
No course on writing prose would be complete without a mention of "point-of-view" (PoV). There isn't much new to say on the issue (other than it's often called "Perspective" nowadays), but being reminded of the possibilities does no harm - it's easy to slip into habits. Writers probably don't try out different PoVs enough, or try different combinations. So first we'll run through the basic options, broaden them out, then look in depth at a particular first-person situation.



\subsection*{1st person}
Example: "I was born in the year 1632, in the city of York, of a good family, though not of that country, my father being a foreigner of Bremen, who settled first at Hull" (1719)


As you can see from the example, it's been around a long while. Henry James in his preface to "The Ambassadors" wrote "the first person, in the long piece, is a form doomed to looseness"  but in 2002 David Lodge believed that "a majority of literary novels published in the last couple of decades have been written in the first person" ("Consciousness and the Novel", p.86). There's a sense of directness. However, from inside one person you see only the outsides of others so the view may be subjective, untrustworthy. Note that


\begin{itemize}
\item The person needn't be the main protagonist - in "The Great Gatsby" for example, the narrator's a minor character, and in the Sherlock stories Dr Watson's usually the narrator. 
\item The narrator may not be very self-revealing. Jeanette Winterson, writing about her "Written on the Body" novel said "the narrator has no name, is assigned no gender, is age unspecified, and highly unreliable. I wanted to see how much information I could leave out - especially the kind of character information that is routine - and still hold a story together". At the other extreme there's stream-of-consciousness.
\item The narrative voice may be very different to the 1st person's voice in dialogue. "Strangely Comforting", a short story by Sadie McKenzie, has a narrator who thinks like this - "Then our eyes meet and my words snag. Vulnerability crouching behind the feral aggression. Me and Kayleigh, united by fierceness masking our fear" but talks like this - "I could be some crazy bitch, just outta pen. I could've just chipped from Holloway. Watch me follow you home an' torch your yard ... Maybe I'll get my brother an' his bredrin to gang you. You up for that?"


\item The first-person PoV can be used in the plural - e.g. "The Virgin Suicides" by Jeffrey Eugenides
\end{itemize}

\subsection*{2nd person} 
Unusual. It's used in "Bright Lights, Big City" by Jay McInerny ("You are not the kind of guy who would be at a place like this at this time of the morning. But here you are, and you cannot say that the terrain is entirely unfamiliar, although the details are fuzzy"). It might sound tedious after a while, though Sarah Hall used it well for "Bees" (a short story in the excellent "The Beautiful Indifference" collection). In Vendela Vida’s novel "The Diver’s Clothes Lie Empty" it's used to depict a malleable self.



If you want to know more, read 


\begin{itemize}
\item Fludernik, who's made a special psychological study of "You"
\item  \url{http://members.westnet.com.au/emmas/2p/thesis/0a.htm} (The Second Person: A Point of View? The Function of the Second-Person Pronoun in Narrative Prose Fiction) by Dennis Schofield - a full thesis).
\item \url{http://www.chuffedbuffbooks.com/writing-in-second-person-atwood-to-tolstoy/} (writing in second person: Atwood to Tolstoy)
\end{itemize}

\subsection*{3rd person}
Example: "In the beginning God created the heavens and the earth." 


A wide range. There's often an attempt to sound objective. It can be


\begin{itemize} 
\item from a particular character's viewpoint ("privileged" "limited") in which case it becomes rather like a first person piece.
\item from a few characters' viewpoints - "Third person dual",etc 
\item omniscient, with the hidden narrator able to see inside the characters
\item omnipresent but not omniscient - objective, like a passive camera.
\end{itemize}

The following table covers most of the alternatives mentioned above

 \begin{tabular}{|l|l|l|l|}\hline
& 1st & 2nd & 3rd\\\hline
stream-of-consciousness &  & & \\\hline
privileged &  & & \\\hline
objective &  & & \\\hline
omniscient &  & & \\\hline
\end{tabular}

How many of the boxes in that table correspond to viable combinations? What features doesn't this table cover? To help answer those questions, let's try an exercise.



\subsection*{Exercise 1: What's the PoV?}
\begin{narrow}{1.0cm}{1.0cm}
PoV isn't always as simple as the above options might make you think. Look at 
the extracts below. Describe the narrators, how much they know, and their 
relationship to "you" or the reader. Try to place the extract in the right 
box of the table above.
\end{narrow}
\begin{enumerate}
\item 
\begin{narrow}{1.0cm}{1.0cm}
Now Alan is putting a dish in the oven. 'Forty-five minutes,' he says, looking 
at his watch. He takes off his apron, hangs it on the back of the door, and 
heads for the living room. 'Seven-fifteen, the highlights. You two stay and 
natter. It'll be ready at eight.'

'Highlights?' says Agnieszka, puzzled, wondering why Alan wants to watch a 
programme about hairdressing.

'The cricket,' says Alan, turning on the TV. 'First day. India all out for 
only 198 and we're already 64 for 1. That's a lot of wickets for the first 
day. England are doing pretty well.' He's almost rubbing his hands. I can 
smell the garlic.
\end{narrow}


\item 
\begin{narrow}{1.0cm}{1.0cm}
The letter that would change everything arrived on a Tuesday. It was
 an ordinary morning in mid-April that smelt of clean washing and
 grass cuttings. Harold Fry sat at the breakfast table, freshly
 shaved, in a clean shirt and tie, with a slice of toast that he
 wasn't eating. He gazed beyond the kitchen window at the clipped
 lawn, which was spiked in the middle by Maureen's telescopic washing
 line, and trapped on all three sides by the neighbours' closeboard
 fencing.

'Harold!' called Maureen above the vacuum cleaner. 'Post!'

He thought he might like to go out, but the only thing to do was mow
 the lawn and he had done that yesterday

...

Upstairs Maureen shut the door of David's room quietly and stood a
 moment breathing him in. She pulled open his blue curtains that she
 closed every night, and checked there was no dust where the hem of
 the net drapes met the windowsill. She polished the silver frame of
 his Cambridge portrait, and the black and white baby photograph
 beside it. She kept the room clean because she was waiting for David
 to come back, and she never knew when that would be. A part of her
 was always waiting. Men had no idea what it was like to be a mother.

\end{narrow}

\item 
\begin{narrow}{1.0cm}{1.0cm}
For some probably economic reason it was usually a woman who was chosen for 
this particular duty, and Grady gave as his motive in selecting Tess that she 
was one of those who best combined strength with quickness in untying, and 
both with staying power, and this may have been true
\end{narrow}

\item 
\begin{narrow}{1.0cm}{1.0cm}

Here is Edward Bear, coming downstairs now, bump, bump, bump, on the back of 
his head, behind Christopher Robin. It is, as far as he knows, the only way 
of coming downstairs, but sometimes he feels that there really is another way,
 if only he could stop bumping for a moment and think of it. And then he 
feels that perhaps there isn't. Anyhow, here he is at the bottom, and ready 
to be introduced to you. Winnie-the-Pooh.

When I first heard his name, I said, just as you are going to say, "But I 
thought he was a boy?"
\end{narrow} 

\item 
\begin{narrow}{1.0cm}{1.0cm}
Let's go to Rupi's brothers now

They've been patient for so long, hidden behind the mouth of the underpass. 
The ferns are cool against their backs. Their soiled clothes blend so well 
against the wall. They are almost as invisible as we are.

Now, this is what they've waited for.

This is the moment I wanted you to see. I hope that you too have a taste for 
the unusual. For the brutal.

If you look now, out of the corner of your eye, maybe you'll see the rest of 
us, sitting on the canal edge, perched upon the roofs, laid out along the top 
of the walls
\end{narrow}

\item 
\begin{narrow}{1.0cm}{1.0cm}
No. I don't care what your agenda is. I'm Miguel. I'm telling you about Pepito. I will have to tell you because he can't, not now, and I think he is important, in his way. So just for a few minutes, still the buzzing, OK?
Pepito's island? Oh, it is beautiful. We always knew that.
\end{narrow}


\item 
\begin{narrow}{1.0cm}{1.0cm}
Strange to say, I expected more from literature than from real, naked
 life. Jan Bronski, whom I had often enough seen kneading my mother's
 flesh, was able to teach me next to nothing. Although I knew that
 this tangle, consisting by turns of Mama and Jan or Matzerath and
 Mama, this knot which sighed, exerted itself, moaned with fatigue,
 and at last fell stickily apart, meant love, Oskar was still
 unwilling to believe that love was love; love itself made him cast
 about for some other love, and yet time and time again he came back
 to tangled love, which he hated until the day when in love he
 practiced it; then he was obliged to defend it in his own eyes as the
 only possible love.
\end{narrow}
\end{enumerate}

\textit{1. 1st person but omniscient too? 2. 3rd person, dual-privileged. Foreshadowing - the narrator knows the plot. 3. The narrator doesn't know everything. 4. The narrator's in complete control, claiming even to know about the reader. 5. Narrator and reader are participants. 6. Narrator hears readers' questions. 7. Changes PoV.}
I hope that these real-life examples ("24 for 3" (Jennie Walker),
"The Unlikely Pilgrimage of Harold Fry" (Rachel Joyce),
"The World of Pooh" (AA Milne),
"Dead Fish" (Adam Marek),
"Tess of the d'Urbervilles" (Thomas Hardy),
"The Carob Tree" (Vanessa Gebbie),
"The Tin Drum", (Gunter Grass)) show that the simple 1st- 2nd- 3rd- person description of PoV needs to be elaborated upon, which is what I'm going to do next.



\subsection*{Extra characters}
Whenever there's an attempt at communicating there's a sender and a receiver even if they're only implied


\subsubsection*{Sender}
\begin{itemize}
\item I mentioned earlier that the "I" needn't be the central character. If the narrator's a character who experiences the events of the story, they're called an "internal narrator", but they might not be in the story at all. They might be the author, or an unnamed character, or a storyteller. Here are some examples -
\begin{itemize}
\item  "Are you sitting comfortably? Now I'll begin".  (BBC)
\item "I should point out I found out about all of this at a much later date. I'm not part of the story yet" (Jim Murdoch)
\item "Now befaw we go too fah down dis road let's you and me get a few tings straight: I'm yer narratah ... whad I say goe" (Jim Murdoch)
\end{itemize}

Sometimes the author butts in (for example, hundreds of pages into Zadie Smith's "NW", there's a surprize "Reader: keep up!"). An anonymous, unreliable narrator is used in a chapter of Ulysses.

\end{itemize}
Tense complicates matters. In a first person story, the main character may also be the narrator, but if it's told in the past tense it might be as if they're 2 different characters. For example, in "A Summer Bird-cage" by Margaret Drabble

 \begin{itemize}
\item The narrator makes brief appearances early on, saying things like "I still remember the way she said that". 
\item Chapter 5 begins "I now find myself compelled to relate a piece of information which I decided to withhold, on the grounds that it was irrelevant, but I realize increasingly that nothing is irrelevant." On the next page it says "It is only now, at the time of writing (or rather, indeed, rewriting) that it occurs to me ...".
\item On p.207 the narrator arrives and stays - "As I sit here, typing this last page". 
\end{itemize}
In total the narrator addresses the reader for about a page, but that's enough to add another layer to the novel. And what about this? 1st or 3rd? "In a village of La Mancha, the name of which I have no desire to call to mind, there lived not long since one of those gentlemen that keep a lance in the lance-rack, an old buckler, a lean hack, and a greyhound for coursing." (1605)



\subsubsection*{Receiver}
The "you" might be one of the characters in the story (especially in epistolary pieces) or it might be the common reader, or it might be more specialised. Italo Calvino in "If on a Winter's Night a Traveler" plays with the options. Even if there's not an explicit "you", there are assumptions about the reader. The "implied addressee" might be an assumed child (in children's literature), a literary reader, etc.



"You" might be addressed to you, the reader (a bit like Miranda Hart in her sit-com). Here's an example -

\begin{narrow}{1.0cm}{1.0cm} 
As soon as I got to Borstal they made me a long-distance cross-country runner. 
I suppose they thought I was just the build for it because I was long and 
skinny for my age ... You might think it a bit rare, having long-distance 
cross-country runners in Borstal but you're wrong, and I'll tell you why"
 ("The Loneliness of the Long Distance Runner", Sillitoe, p.7)
\end{narrow}

Sometimes narrators act as if they're being asked questions by someone who's never identified - as in the Gebbie example earlier.

 

\subsection*{Involvement with the story}
Sometimes readers find it difficult to identify with any specific character when the viewpoint's omniscient. If readers aren't directly addressed they might be tempted to identify with one of the characters, but who? The only one we can see inside of? The only morally sound character?  What PoV maximizes immersion? Such questions can be answered experimentally, a fertile area of research. Using EEG and fMRI researchers have measured effects. Sometimes they use erotica, but we'll gloss over that. I think that the conclusion is that there might be an initial effect when using the 2nd person which is why PR people often use it - "1 in 4 people who smoke more than 20 cigarettes a day die of cancer" is less effective than "Hey, do you smoke more than 20 a day? Come on, be honest. If your answer's yes you've a 1 on 4 chance of dying of cancer". Readers soon get used to this trick though. Other factors matter at least as much.




\subsection*{Mixing things up}
Different sections of a story can have different points-of-view. It's not uncommon to have "I" being a different person in alternate sections or chapters (see for example "The Book of Human Skin" by Michelle Lovric, "Gone Girl" by Gillian Flynn, or part of Sartre's trilogy).


 Or the 3rd-person narrative may take on a flavour of the voice and opinions of the character concerned, in free indirect style. Here are examples of speech modes from \url{http://en.wikipedia.org/wiki/Free_indirect_speech} (Wikipedia)


\begin{itemize}
\item \textit{Direct speech} - He laid down his bundle and thought of his misfortune. "And just what pleasure have I found, since I came into this world?" he asked.
\item \textit{Reported speech} - He laid down his bundle and thought of his misfortune. He asked himself what pleasure he had found since he came into the world.
\item \textit{Free indirect speech} - He laid down his bundle and thought of his misfortune. And just what pleasure had he found, since he came into this world?
\end{itemize}
Here's another example of \textit{Free indirect speech}, adapted from one by Emma Darwin -  
Emily was one of those people who hates confronting liars. She put down her coffee. What a bastard he was! He was obviously lying. She picked up her coffee again and said how sweet it was of John to be so helpful. 
Note the transition from narrator voice to the character's voice and back again. As Emma Darwin says, it helps turn \textit{tell} into \textit{show}. Jane Austen uses it.



From 3rd person the voice can switch from one limited viewpoint to another. Or there can be long monologues or flashbacks (whodunits, with long witness-PoV sections, have put this device into the mainstream). Harry Potter novels dip into various minds in limited third person sections, I think. The narrative can zoom in ever further, giving us a character's interior monologue and physical perceptions as they occur. 


Near the start of "Moon Tiger" by Penelope Lively (Penguin, 1988) we read that "The voice of history, of course, is composite. Many voices; all the voices that have managed to get themselves heard. Some louder than others, naturally ... So, since my story is also theirs, they too must speak - Mother, Gordon, Jasper ... Except that of course I have the last word. The historian's privilege". The PoV-switching continues throughout the book. Sometimes a paragraph is followed by a paragraph describing the same events from a different viewpoint. The viewpoint may differ slightly (both 3rd person privileged, but from 2 people's viewpoints) or the viewpoints might be 1st and 3rd person.
\begin{narrow}{1.0cm}{1.0cm}  
And over there if I am not mistaken is this chap who might wangle me a
 ride up to the front if I play it right She smiles - the glossy
 lipsticked smile of the times. She approaches his table - a neat
 figure in white linen, bright coppery hair, high-heeled red sandals,
 bare sunburned legs - and he rises, pulls out a chair, clicks his
 fingers at the suffragi.  And looks appreciatively at the legs, the
 hair, the outfit which is not the get-up of the average woman press
 correspondent.

At least it is to be assumed that that is what he was doing since he
 tried later to get me into bed
 (p.69)
\end{narrow}






In "Zennor in Darkness" by Helen Dunmore is an example of a novel where point-of-view is fluid, changing from person to person rapidly, and varying in the depth of stream-of-consciousness.


\begin{itemize}
\item On p.179 there's this paragraph that begins with Clare's PoV, 3rd-person privileged, then becomes 1st person - "Clare turns her face into the pillow and grips it. Hannah hinted at something in Sam's letter; something which had happened to Sam, and changed him. There must be things I don't know and can't begin to imagine"

\item Chapter 22 begins with Clare's PoV, 3rd person, when she was 5. After a few pages it becomes omniscient for a page. Then there's "Nan sews. Her thoughts are like stitches, each one tiny and precise and not much in itself, but making up the strong seam" followed by Nan's introspection, then "Fifteen years later Nan sews and thinks of Clare", the introspection continuing. Then suddenly we seeing things from Clare's 20 year-old 1st person PoV.

\end{itemize}

In "In a strange room", Damon Galgut switches from 3rd to 1st person within a paragraph, the same person being the subject - e.g.

 \begin{narrow}{1.0cm}{1.0cm}
The figure is a man about his own age, dressed entirely in black. Black pants 
and shirt, black boots. Even his rucksack is black. What the first man is 
wearing I don't know, I forget
\end{narrow}
(p.3)


According to one reviewer it shows that "The narrator is both involved and distant". See Francine Prose's "Reading like a writer" for other examples.



But what about "inconsistent PoV"? The change in PoV needn't be extreme to be distracting. What about these?


\begin{itemize}
\item Frogs could be heard croaking as we neared the pond
\item When one feels tired, a cake will give you quick energy
\end{itemize}
Or suppose in a first person piece you read "She felt ashamed". You might say "hey, hold on, how does the narrator know this?", but even famous authors (Dickens etc) can be inconsistent. Gunter Grass in the example earlier changes PoV between one sentence and the next. Hemingway in his much praised "Hills Like White Elephants" twice breaks the "rule" of objectivity. If you can surprize readers with a plot-twist what's wrong with surprising them with a narrative switch?





\subsection*{Other media}
\begin{itemize}
\item Media Studies has taken over from Eng Lit in some places. The courses overlap when it comes to PoV. The default for film is "3rd person objective". Can you think of some interesting alternatives? There's split-screen (which I've tried in prose); subtitles used as in "Annie Hall" to show an alternative, simultaneous PoV; "The Diving Bell and the Butterfly"; voice-overs; jerky, hand-held scenes in "The Blair Witch Project". How about "What Maisie Knew" (2012)?
\item There are first-person shoot'em'up computer games too. Sometimes you can choose your viewpoint. Perhaps e-novels should let you do that too.
\item You can also get ideas from paintings - Cezanne's multiple viewpoints, Breugel's multiple-narrative street scenes
\end{itemize}

\subsection*{Theories}
As you can see, things are getting complicated, which is why straightforward "PoV" is out of fashion. A 1st person and a 3rd person story may have more in common than two 3rd person stories have. As well as knowing who's doing the seeing we need to identify who's doing the narrating, where they are in relation to the story-space and how much they know about the characters and plot.

 

People have tried organising the possibilities into a diagram or table. It can help identify untapped possibilities, or show how certain stories resemble each other


Stanzel decided to consider 3 main aspects


\begin{itemize}
\item MODE - narrator ... reflector (telling a story or manipulating?)
\item PERSPECTIVE - internal ... external ... authorial
\item PERSON - 1st person ... identity ... non-identity
\end{itemize}
\includegraphics{stanzel.png}

If you had 2 of these you could construct a table - maybe with "1st", "2nd", "3rd" as column headings, and "internal", external" as the rows. It would be interesting to think of novels that would fit in the cells of the table. But what can you do with 3 factors? A 3D table? Stanzel decided to have each aspect as a diameter of a circle - "Stanzel's typological circle". People elaborated this until it became rather occult.



Genette sliced things up another way, using "focalisation" (omniscient; 1 person, "camera") and voice (1st person, etc) and considered these to be independent features. It seems fair enough to me. Let's see if some diagrams help

This is a standard "objective" scenario, like a film. There's an invented world within which the action unfolds, and a passive camera is there to record events, unable to see inside people, though close-ups are possible.

\includegraphics[width=6cm]{storyworldpersoncamera4.png}

In this alternative there are dotted lines around the person to denote a see-through boundary; 3rd-person privileged. Other characters could be added - some with see-through boundaries, some with solid boundaries.

\includegraphics[width=6cm]{storyworldpersoncamera3.png}

Let's move the camera to make it 1st-person. We'd better keep the person's boundary see-through

\includegraphics[width=6cm]{storyworldpersoncamera5.png}

Now let's move it so that the narrator's outside the story, able to see the plot from various angles - the storytelling scenario. But there's still an assumption that the storyworld is stable, that a passive camera records the unfolding events.

\includegraphics[width=6cm]{storyworldpersoncamera2.png}

But sometimes the narrator doesn't want us to believe in the storyworld - it's all a game inside the narrator's head - "The French Lieutenant's Woman" perhaps. The narrator will survive even if the story is abandoned, restarted, redrafted, etc.

\includegraphics[width=6cm]{storyworldpersoncamera6.png}

\subsection*{How to choose a PoV}
There are many factors to consider. For example


\begin{itemize}
\item In what way is the PoV you're choosing going to limit you? 
\item It's easier for readers to assume that the narrator is the author. If you're male and the main character's female, is 1st person too risky?
\end{itemize}
The good news is that it's never too late to change. And of course you can keep various versions on file. Alice Munro often revises her stories between their original publication in a periodical and the republication in a collection. For example, she frequently writes a story from both the third-person and first-person point of view before deciding which to use in the final version. Changing from 1st to 3rd person might not be a big deal. Changing from "objective" to "omniscient" is another matter.




\subsection*{Exercise 2 - choosing a PoV}
\begin{narrow}{1.0cm}{1.0cm}
You're settling in to your new student room. There's a knock at the door. 
Someone cute asks if you have any sugar. You offer them in for a tea. Narrate 
until the moment the tea is poured. Do it twice, from different perspectives. 
Pick the option most natural to you, then one that's challenging.
\end{narrow}

\subsection*{Special first-person PoVs}

I'd like now to focus on particular 1st person PoVs because they're increasingly popular. The narrator can be dead - there's "The Book Thief" by Markus Zusak, where the narrator is Death, and "The Lovely Bones" by Alice Sebold, where a young girl, having been killed, observes, from some after-life vantage point. Narrators can be mad, or animals (Ian McEwan had a pet ape as a narrator).




Or the main character can be a child. Writing from a child's point-of-view isn't easy. Done well though, it can be effective and affecting, so that's what I'm going to focus on.

\includegraphics[width=5cm]{childrenfilm.jpg}

In his film, "A story of children and film", Mark Cousins suggests that children's been dealt with in film more often than in novels. Maybe so, though novels are catching up -  a judge of the 2013 Man Booker competition said there were a lot of child-PoV entries.
 Most stories of this type use a third-person-priviledged point-of-view, though a first-person treatment is possible. Some people (me included) rarely produce child-centred stories, which is odd - after all, we were all children once. Two story collections I've read have a fair proportion of child-centred stories, so I thought I'd bring the authors' views into the discussion.
Some writers raid their own pasts
\begin{itemize}
\item "The key, I think is memory: quite simply, remembering, never forgetting what it was like to be a child - ... when I was in my early twenties I made a conscious vow ... never to forget what it was like to be a child ... But I do also happen to have a very good memory: ... 'Leaf Memory' is based on a real-life memory of my own, aged two years and two months" - Elizabeth Baines

\end{itemize}
\subsection*{Exercise 3}
\begin{narrow}{1.0cm}{1.0cm}
Write 100 words about your earliest memory. Do it twice - first in a language closest to the way you thought at the time, then using your current powers of expression.  
\end{narrow}


My memory's nowhere near that good. I'm a parent, which you'd have thought should be useful in this context, but childlessness may have advantages. Parents have less time to write, but that's not all - in "The Psychologist" March 2009 they reported on a survey that found that parents are no happier than childless couples. In fact, once the children leave home, parents are sadder. One begins to wonder what the point of children is.



\begin{itemize}
\item "having no children myself means that I've never fully grown up. I'm at the age where many of my friends are wondering why hostile, sulky delinquents from outer space have occupied their teenage children's bodies. And what do I do? Easy, I side with the kids. ... Basically, I can't grasp the crisis from the parent's viewpoint, however hard I try" - Charles Lambert
\end{itemize}

Writing's hard enough as it is without burdening oneself with extra handicaps, so why should authors restrict themselves to a child's viewpoint and vocabulary? It's fair enough in children's fiction but what about fiction for adults? Let's look at each restriction in isolation



\begin{itemize}
\item \textit{Viewpoint} -  Though children might not understand what's
going on, and might be unable to be involved in the scene, they have
certain advantages as observers - like cameras, they might see things
from a new angle and
might be ignored by the protagonists. The child might not understand
what's going on, but readers are likely to. The difference between
the character's and  the reader's understanding can be
exploited for laughs or for more serious effect.
On the BBC's web-site they give
the example of this - a child bursting into his parents' bedroom, upset to find
them wrestling naked on the bed. Successful writers consciously exploit this
  irony
\begin{itemize}
\item 
"children can have instinctual knowledge which we adults can lose, and these insights yet gaps can be the stuff of dramatic conflict and motor a story" - Elizabeth Baines

\item 
"one of the things I'm doing when I choose to use children as the channel through which the narrative is seen is what Henry James did with Maisie; I'm exploiting their clear-sightedness and innocence. Children see everything, but don't necessarily understand any of it. Whether they're protagonists or witnesses, they tend to be one step behind - or to one side of - the attentive adult reader, which sets up an interesting narrative gap through which the unsettling elements can squeeze." - Charles Lambert

\end{itemize}



\item \textit{Language} -  Children may not have a wide, intellectual
vocabulary, but that needn't be such a restriction. They can be original 
in their use of words, less restricted by convention and social mores.
\end{itemize}
A way round both of these limitations is to use a fluid 3rd-person priviledged point-of-view, rather as in the Joyce example below. Alternatively, if it's written in the past tense, the narrator can gloss over these difficulties (the author can create an adult character who recalls an amazing amount about childhood), though it dilutes the effect.



\subsection*{Exercise 4: Guess the age!}
How young can you go? "My Mother's Dream" (Alice Munro) is from the viewpoint of someone before their conception, then as a foetus, then a baby for most of the story, which is probably beyond the call of duty. Try guessing the age of the children in these extracts, and the supposed age of the narrator.
\begin{itemize}
\item 
\begin{narrow}{1.0cm}{1.0cm}
Maisie received in petrification the full force of her mother's
 huge and painted eyes - they were like Japanese lanterns swinging under festival arches
\end{narrow}


\item 
\begin{narrow}{1.0cm}{1.0cm}
After a while of playing, Mary gets bored and speaks on the phone. She always 
twirls the cord around her finger and gets her whole body wrapped up in it. 
It’s silly. Sometimes I don’t think she’s really a grownup. Maybe she’s just 
playing dress-ups.

Daddy walks in with a big smile on his face, and Mary skips up to him like a 
little girl and gives him a kiss on the cheek.

“Did you go to see your mother today?” That’s Daddy speaking to Mary, not me. 
She nods and does doll’s eyes and hangs her head to the side making a stupid 
groaning sound. She sounds like my Ted in the mornings. 
\end{narrow}



\item 
\begin{narrow}{1.0cm}{1.0cm}
Once upon a time and a very good time it was there was a moocow coming
down along the road and this moocow that was coming down along the road
met a nicens little boy named baby tuckoo...

His father told him that story: his father looked at him through a
glass: he had a hairy face.

He was baby tuckoo. The moocow came down the road where Betty Byrne
lived: she sold lemon platt.

    O, the wild rose blossoms
    On the little green place.

He sang that song. That was his song.

    O, the green wothe botheth.

When you wet the bed first it is warm then it gets cold. His mother put
on the oilsheet. That had the queer smell.
\end{narrow}



\item 
\begin{narrow}{1.0cm}{1.0cm}
We were coming down our road. Kevin stopped at a gate and bashed it with his stick. It was Misses Quigley's gate; she was always looking out the window but she never did anything.
- Quigley! 
- Quigley! 
- Quigley Quigley Quigley!
Liam and Aidan turned down their cul-de-sac. We said nothing; they said nothing. Liam and Aidan had a dead mother. Missus O'Connell was her name.
- It'd be brilliant wouldn't it? I said
\end{narrow}





\item 
\begin{narrow}{1.0cm}{1.0cm}
The wide playgrounds were swarming with boys. All were shouting and the
prefects urged them on with strong cries. The evening air was pale and
chilly and after every charge and thud of the footballers the greasy
leather orb flew like a heavy bird through the grey light. He kept on
the fringe of his line, out of sight of his prefect, out of the reach
of the rude feet, feigning to run now and then. He felt his body small
and weak amid the throng of the players and his eyes were weak and
watery.
\end{narrow}



\item
\begin{narrow}{1.0cm}{1.0cm}
MARCH 25, MORNING. A troubled night of dreams. Want to get them off my
chest.
\end{narrow}



\item 
\begin{narrow}{1.0cm}{1.0cm}
Saturday January 3rd
I shall go mad through lack of sleep! My father has banned the dog from the 
house so it barked outside my window all night. Just my luck! 
My father shouted a swear-word at it. If he's not careful he will get done by
 the police for obscene language.
\end{narrow}

\item
\begin{narrow}{1.0cm}{1.0cm}
'Is that your name?' I was bold enough to ask the Miss more prone to mirth.
'Eleanor is what I was christened but people call me Ellie.'
Idly, I said, 'That's not what my mother calls you.'
'What does she call her?' enquired the one who was not Ellie.
'Not just her, both of you.'
What does she call us?' 
Her dress smelled of corridor.
The sisters awaited an answer. Ellie, dried dribbles of Wall's ice-cream on 
her frock, seemed as eager to know as her sister.
A curler in her hair, a clip between her teeth, my mother held her breath,
'She calls you …' I paused to accord the phrase the respect with which I had 
always heard it uttered, 'She calls you "the Misses Linster".'
Though obviously the cause of amusement, I wasn't sure what was funny. 
\end{narrow}



\end{itemize}


The extracts are from "What Maisie Knew" (Henry James; the child's 7),
"The Book" (Jessica Bell; the child's 5),
"A Portrait of the Artist as a Young Man" (James Joyce),
"Paddy Clarke Ha Ha Ha" (Roddy Doyle; the boy's 10),
"A Portrait of the Artist as a Young Man" (James Joyce),
"A Portrait of the Artist as a Young Man" (James Joyce),
"The Secret Diary of Adrian Mole, aged 13 3/4"  (Sue Townsend) and
a story by Ian Madden; the child's 6 years old.
Authors often seem to have over- or under-estimated the
child, but kids have an irritating habit of not acting their age -
one
moment they talk like an adult, next moment they sulk like a baby. In any case, one shouldn't expect dialogue in literature to be like Real Life - it has to be artificial to some extent but how much? It can be difficult to convince the reader of the narrator's age.



\subsection*{Examples}
Let's see how the experts do it


\begin{itemize}
\item 
\begin{narrow}{1.0cm}{1.0cm}
Today I'm five. I was four last night going to sleep in Wardrobe, but when I 
wake up in Bed in the dark I'm changed to five, abracadabra. Before that I was 
three, then two, then one, then zero.
"Was I minus numbers?"
\end{narrow}


\item 
\begin{narrow}{1.0cm}{1.0cm}
A shadow made me start as my mother's face loomed towards me where I lay, eight months old, tongue-tied, spastic and flailing on my course rug, on the warm lawn, in the summer of 1947 - in an English country garden. My father was playing French cricket with Miranda and John, and I could hear a tennis ball

\end{narrow}




\item 
\begin{narrow}{1.0cm}{1.0cm}
Early one morning as we were beginning our day’s play in the back yard, Jem and I
heard something next door in Miss Rachel Haverford’s collard patch. We went to the
wire fence to see if there was a puppy—Miss Rachel’s rat terrier was expecting—
instead we found someone sitting looking at us. Sitting down, he wasn’t much higher
than the collards. We stared at him until he spoke:
“Hey.”
“Hey yourself,” said Jem pleasantly.
“I’m Charles Baker Harris,” he said. “I can read.”
“So what?” I said.
“I just thought you’d like to know I can read. You got anything needs readin‘ I can do
it...”
“How old are you,” asked Jem, “four-and-a-half?”
“Goin‘ on seven.”
“Shoot no wonder, then,” said Jem, jerking his thumb at me. “Scout yonder’s been
readin‘ ever since she was born, and she ain’t even started to school yet. You look right
puny for goin’ on seven.”
“I’m little but I’m old,” he said.

\end{narrow}

\end{itemize}

Note that these extracts tell the reader the age of the child. Kids tell each other their ages, so it's not too artificial. It gets round one problem when writing such pieces.








\subsection*{Problems and criticisms}
\begin{itemize}


\item It's very tempting to slip out of character for a few paragraphs. 
A commonly used way to include an adult's viewpoint is for the child to be an uncomprehending messenger - e.g. to have the child find an adult's diary and read it (Paula Sharp calls that a hackneyed device though!). Here's Elizabeth Baines' approach
\begin{itemize}
\item  "The story 'Power' ... strictly, use[s] a child narrator, ie, the voice is that of the child as a child, and in this case in the present tense, as the story is happening. This is the most restrictive way of adopting a child's viewpoint, with least chance for authorial intervention. The main way I get round the restriction here is to splice the child's narrative with the mother's phone calls on which the child eavesdrops."
\item  "In 'Star Things' ... the child is constantly and innocently quoting things her parents have said"
\end{itemize}
She notes however that "the children's voices in these stories aren't entirely
  naturalistic, I do take linguistic licence, as they're not intended as
  straightforward dramatic monologues"

\item Even the best adult books with child narrators
risk being treated as if they're children's books

\item One has to be rather careful about using material that can be traced back
  to a particular child - moreso than with consenting adults.
\end{itemize}

\subsection*{Special Needs}

Authors have tried combining age limitations with other features - in particular, cleverness.
 In a sense, these writers are having it both ways; they can exploit the freedom and freshness of the child narrator without having to make too many compromises regarding vocabulary or intellect.



\begin{itemize}
\item "The Curious Incident of the Dog in the Night-time" (Mark
  Haddon) has a clever, 
  autistic 15-year-old narrator
\item "Extremely Loud and Incredibly Close" (Jonathan Safran Foer) 
 is held together by Oskar, a precocious and obsessive nine-year-old
 polymath
\item "How the Light Gets In" (M.J. Hyland) has a highly intelligent,
  damaged 16-year-old
\item "Flowers for Algernon" (Daniel Keyes) doesn't have a child
      narrator, but the IQ and language of the narrator change in the
      course
of the novel.
\end{itemize}
\subsection*{Exercise 5}
\begin{narrow}{1.0cm}{1.0cm}

Describe an adult situation from a child's PoV. The child can be special if you wish.
\end{narrow}



\subsection*{Conclusions}
\begin{itemize}
\item Try things out! Write multiple versions. Mix versions.
\item Don't be dogmatic.
\item Watch films and play computer games.
\end{itemize}

\subsection*{Discussion Points}
\begin{itemize}
\item Are child-narrator stories usually autobiographical?
\item What other devices do authors use to bring an adult perspective into child-narrator stories?
\item What 1st person child narrator novels/stories have you read? Did they work?
\end{itemize}


\subsection*{Authors quoted}

The quotes are used (with the authors' permission) from Virtual Booktours that they made - Elizabeth Baines' "Around the Edges of the World" Tour and Charles Lambert's "Something Rich and Strange" Tour


\begin{itemize}
\item \url{http://elizabethbaines.blogspot.com/} (Elizabeth Baines) won 3rd prize in the Raymond Carver Short Story Competition 2008. Her book, "Balancing on the Edge of the World" (Salt) was shortlisted for the 2008 The Salt Frank O'Connor Prize.
\item \url{http://charles-lambert.blogspot.com/} (Charles Lambert) was an O.Henry Prize winner in 2007, along with William
  Trevor and Alice Munro. Books include "Little Monsters" (Picador) and "The
  Scent of Cinnamon" (Salt)
\end{itemize}

\subsection*{Examples}
\begin{itemize}
\item  Joyce's "A Portrait of the Artists as a young man"
\item  Hugo Hamilton's "The Speckled People" - people have said "The world here is viewed through the eyes of a child who does not judge, merely details and describes." .... "Though Hugo matures as the story unfolds, the simple, declarative sentences of a child's confused and partial understanding do not. (...) He has made an attempt on something impossible - to show from a child's point of view what a child can't see. To the degree that he succeeds, it's remarkable."
\item Paula Sharp's "Crows over a Wheatfield" - people have said that "the characters are so involving - not since 'To Kill a Mockingbird' or the opening chapters of 'Jane Eyre' has there been a more acute and astute child's view of the world".

\item Sue Townsend's "The Secret Diary of Adrian Mole, Aged 13 3/4". Comedy.
\item Colum McCann's "Everything in this country must" - this
      collection's  stories have 1st person narrators in their early teens.
\item Roddy Doyle's "Paddy Clarke Ha Ha Ha" - has a 10 year old 1st person narrator
\item Daisy Ashford's \url{http://www.stonesoup.com/ash2/ash1.html} (The Young
    Visiters) was written for adults by a 9 year old
\item Harper Lee's "To Kill a Mockingbird" - has a 6 year old narrator.
\item Emma Donoghue's "The Room" - has a 5 year old narrator.
\item  Jessica Bell’s "The Book" - has a 5 year old narrator. See \url{http://jim-murdoch.blogspot.co.uk/2013/03/the-book.html} (Jim Murdoch's blog) for details.
\item "The Life of Pi"? "The Tin Drum"? "Empire of the Sun"?
\end{itemize}





\subsection*{See also}
\begin{itemize}
\item \url{http://emmadarwin.typepad.com/thisitchofwriting/2014/11/ten-ways-to-move-point-of-view-and-dont-let-the-self-appointed-experts-tell-you-otherwise.html} (Ten Ways To Move Point-Of-View (And Don't Let The Self-Appointed Experts Tell You Otherwise)) (Emma Darwin)
\item \url{http://litrefsarticles.blogspot.co.uk/2011/01/multiple-points-of-view.html} (Multiple points-of-view)
\item \url{http://litrefsarticles.blogspot.co.uk/2011/01/child-narrators-in-adult-fiction.html} (Child narrators in adult fiction)
\item \url{http://emmadarwin.typepad.com/thisitchofwriting/2011/10/point-of-view-narrators-1-the-basics.html} (The basics) (Emma Darwin)

\item \url{http://emmadarwin.typepad.com/thisitchofwriting/2011/10/point-of-view-narrators-2-internal-narrators.html} (Internal narrators) (Emma Darwin)
\item \url{http://emmadarwin.typepad.com/thisitchofwriting/2011/10/point-of-view-narrators-3-external-narrators.html} (External narrators) (Emma Darwin)
\item \url{http://emmadarwin.typepad.com/thisitchofwriting/2011/10/point-of-view-narrators-4-moving-point-of-view-and-other-stories.html} (Moving point of view) (Emma Darwin)
\item \url{http://emmadarwin.typepad.com/thisitchofwriting/2013/09/free-indirect-style-what-it-is-and-how-to-use-it.html} (Free indirect style and how to use it) (Emma Darwin)
\item \url{http://en.wikipedia.org/wiki/Narrative_mode} (Narrative mode) (Wikipedia)
\end{itemize}

\newpage
\section{Rewriting}

Congratulations! You've got further than many budding writers - you've actually written something. Pat yourself on the back. But you know there's more work ahead. How much more depends on you.


People have different attitudes to rewriting. For some people it's checking for typos and
removing some superfluous words. That's certainly part of the task (often called ``polishing''), and we'll do a few
exercises on that topic, but that's the easy bit.

We'll also be looking at other changes we can make, and the inhibitions that stop
us making them. A particular problem is over-familiarity with the text, and - let's be
honest - boredom having to go over the same old stuff again and again.

During the re-write you may discover that your novel
isn't ever going to work. I'll look at what to do in that situation too, because all is not lost.

I'll mostly talk about novels, though I'll mention other genres in passing. I'm going to throw lots of suggestions to you, some of them contradictory. Just pick the ones that suit you!

\subsection*{Exercise 1}
Re-write these -
\begin{enumerate}
\item They are so gripped by the film, they're frozen in time and sit on the sofa like statues, absolutely still and hardly breathing. (from a novel)


\item In the not too distant future, college freshmen must all become aware of the fact that there is a need for them to make contact with an academic adviser concerning the matter of a major. (from a prospectus)

\item I stepped out of the city and into the park. It was as simple as that.

It was January, it was a foggy day in London town, I'd got off the Tube at Great Portland Street and come up and out into the dark of the day, I was on my way to an urgent meeting about funding. It was possible in the current climate that funding was going to be withdrawn so we were having to have an urgent meeting urgently to decide on the right kind of rhetoric. This would ensure the right developmental strategy which would in turn ensure that funding wouldn't conclude in this way at this time. (from a short story)
\end{enumerate}



\subsection*{Attitudes to re-writing}
\subsubsection*{First draft?}
Do you try to produce a final version at the first attempt? Do you perfect each chapter before going on to the next? There's no right and wrong amongst writers or song-writers. Some barely rewrite -
\begin{itemize}
\item Elton John generally sat with a page of Bernie Taupin's lyrics. If he didn't
finish in 30 minutes he gave up.

\item James Tate (US poet) claimed that he never revised. Instead, he would write a promising line and sit and wait for the next good line to come to him.

\item Anthony Burgess claimed that he wrote a chapter, revised it, then wrote the next chapter and revised it until he finished the book. 
\end{itemize}

Others rewrite obsessively -
\begin{itemize}
\item Ernest Hemingway  used to brag that he rewrote the ending to "A Farewell to Arms" 39 times. He said that when you threw away stuff that other writers would use, you're doing well.

\item Paul McCartney would sometimes try out his latest composition on anyone too polite to refuse. Coming back from Yorkshire once, he stopped at a village pub in Bedfordshire and tried out a version of "Hey Jude".
\end{itemize}

Some people approach successive drafts in different ways.
\begin{itemize}
\item  ``You write a script twice. The first time you pour out all your passion, anger, energy, and frustration. Then you go back and write it with your head'' - Jimmy McGovern (TV scriptwriter)

\item  ``The first draft is like a romance, flowering and easy during its first few months until eventually the situation has to normalise, to settle. The second draft is like a marriage. The passion will not have evaporated but it’s now tempered by responsibility" - Ashley Stokes
\item  ``I write every paragraph four times - once to get my meaning down, once to put in anything I left out, once to take out everything that seems unnecessary, and once to make the whole thing sound as if I only just thought of it'' - Marjorie Allingham 
\end{itemize}
For most of us, rewriting's inevitable. We're always re-writing.

%\includegraphics[width=10cm]{4times.jpg}


\subsubsection*{Final version?}
Work is never finished, only abandoned. Nobel prizewinner Alice Munro revises even after publication in a magazine. Before republication in a collection she might write a story from both the third-person and first-person point of view before deciding which to use in the final version. And Auden 
rewrote (or disowned) poems, to the irritation of his publishers.

Maybe you thought you'd finished, but the editor suggests changes. I've twice
turned down magazine editor's suggestions lately.


\subsection*{Alternatives to DIY}
Before we look at rewriting, let's consider alternatives. You could
find a trusted person to swap drafts with. Alternatively 
there are commercial services. One example is \url{https://www.writersandartists.co.uk/writers/services}. They offer several options, amongst them \textit{Editing Services: Final Polish} for £680 – £1020 which is, I presume, the going rate, and an indication of how much work is involved. So think hard before asking someone to look through your work as a favour.


\subsection*{Re-writers block}
Writer's Block is when you're staring at an empty page and nothing happens. Re-writer's Block is when you stare at a full page and nothing happens.

Many of the factors involved with Writer's Block still apply to Re-writer's Block. Indeed, re-writing may itself be a form of Writer's Block. One theory is that writer's block is the subconscious's way of doing you a favour. Perhaps it knows that if you send a work off it will only end in humiliating failure. It also knows that if you're ordered not to send anything away, you'll do just the opposite, so instead it encourages you to re-write. While you're re-writing you won't be sending anything off. And you can re-write for ever.

% Indeed, if your  sub-conscious really is trying to sabotage you, it may try even harder now that time's running out.

%To complicate matters, re-writing may itself be a form of Writer's Block. Here's what Victoria Nelson thinks -  ``As a delaying tactic that keeps the writer from embarking on a new adventure, obsessive rewriting is a highly effective manifestation of the block ... by clinging to the relics of his writing past, he is traveling in the opposite direction from artistic growth.''


What factors particularly inhibit re-writing?
\begin{itemize}
\item Fear of destroying the freshness of the original (``The unconscious creates, the ego edits'' - Stanley Kunitz)
\item Fear that you might realise it's all rubbish, that you've wasted your life
\item Fear of commitment, that there's no going back
\item Rewriting's boring
\item Not knowing where to start
\item Not knowing what to change
\item Self-imposed restrictions - e.g. ``write about what you know''; ``but that's what actually happened!''; sticking to the same length, genre, era, age/gender of characters
\end{itemize}
Many of these are more "project management" than artistic issues. Let's see if we can remove some of these hindrances

\subsection*{Have a plan}
Don't be open-ended. Be focused. Don't fiddle about or tinker - that's a displacement mechanism.

\begin{itemize}
\item If it's rubbish, dump it. What's worse than spending 5 years writing a rubbish novel? Answer: Spending 10 years writing it. So put it away in a drawer. With any luck you'll be able to publish it after you've had another novel published (the first two novels that Iain Banks wrote were published well after his breakthrough novel "The Wasp Factory" came out).

\item Focus your attention on the sections with the biggest pay-off - the start, the end, the crisis moment. Editors will do that, so you might as well too. Chekov suggested that you should write the beginning, middle, and end, then cut the beginning and the end. Elsewhere he suggested that you throw away the first 3 pages. That sounds extreme, but it's worth at least underlining the first interesting sentence in your piece. If it's not near the beginning, have a good excuse ready. 

\textbf{Exercise 2}

Here are 2 beginnings of stories by James Runcie about Grantchester. The first is rather flat. The second is rather poetical. Swap the styles -
\begin{itemize}
\item    Sidney was uneasy. He knew that it was one of his principal duties as a priest to keep cheerful at all times and he liked to think that he was content with his lot in life, but the copy of The Times that he was reading one late April morning in 1963 carried a biblical quotation at the top of the Personal Column that gave him pause

    'Woe unto you, when all men shall speak well of you'

    (\textit{'Death by Water'})

\item As the afternoon light faded over the village of Grantchester, the parishioners lit fires, drew curtains and bolted their doors against the dangers of darkness. The external blackness was a memento mori, a nocturnal harbinger of that sombre country from which no traveller returns. Canon Sidney Chambers, however, felt no fear. He liked a winter's night.

    It was the 8th of January 1955. The distant town of Cambridge looked almost two-dimensional under the moon's wily enchantment and the silhouettes of college buildings were etched against the darkening sky like illustrations for a children's fairytale.

    (\textit{'The Perils of the Night'})

\end{itemize}

\item  For novels especially, you needn't make all the changes as you go along - just make a list; e.g.  - 'beef up X's character'; 'sort out lost-letter plot'; 'revise Chapter Six'; 'check geography of Manchester chapter' (Emma Darwin); 'research into what rubbish is dropped onto pavements'; 'find names of nail-varnish colours' (Tim Love). 

\item How do you decide what needs changing?  Read other's stories critically to practice finding problems, or go through a checklist of features a short story should have. Many checklists are online.

\item You could try to perfect your novel a chapter at a time, as if each one was a short story - see later.

\item Try to add variety to your rewriting sessions. Try to make them fun!

\item Get in the right mood - maybe start the day by revising the previous evening's work? Work in a different room when you revise? Use rituals?
\end{itemize}

\subsection*{Version Control}

\includegraphics[width=2cm,angle=270]{versions.png}

Amongst the inhibitors to change is the fear that you'll make things worse.  

You can work in such a way that you can save all versions and compare old and new versions side by side. Knowing that you can go back to older versions helps make you more relaxed and radical about changing your current version. You're free to experiment.

There are other reasons for preserving versions. e.g for Flash, I try places which have limits of exactly 75, max 100,  exactly 100, exactly 101, max 200, max 250, max 300, max 360, etc, so I have multiple versions of some pieces.

How do you do this? First, if your Word processor allows you to track changes and revisions, do that. Some versions of Office let you recover earlier drafts, some don't. Or you can use Overleaf, GoogleDocs, etc.

%https://tr1.cbsistatic.com/hub/i/r/2014/05/29/d1125349-1185-4546-a869-e4266c696c12/resize/770x/06b177bca7087b6f7afb051aefc32e39/2014152-figc.jpg
\includegraphics{2014152-figc.jpg}

\subsection*{Distancing}
\begin{itemize}
\item \textit{O wad some Power the giftie gie us, to see oursels as ithers see us!}
\item \textit{editing is like sex. If you do it to yourself you can't really call it editing} - Matthew Welton
\end{itemize}

Whatever your attitude to re-writes it can help to distance yourself from the text before attempting a re-write. By ``distancing'' I mean being more detached from the piece, as if it weren't yours, so you can see it as others see it and you're less inhibited about making changes. This distancing can be achieved by waiting, but if time is at a premium there are other options.
\begin{itemize}
\item Samuel Beckett started writing in French to distance himself from his work.

\item ``No passion in the world is greater than the passion to alter someone else's draft'', said H.G. Wells. If you're that type of person, then pretend you're someone else when you read your piece.
\item Write a review of it, or a blurb.
\item Try printing it out in a different font, or reading it out, or recording it and playing it back. 
\end{itemize}

Or you can alternate between small- and large-scale views.


\subsection*{Small scale}
Re-writing your work is little like a surgeon operating on a loved one. It helps to be detached, to operate on the knee, not the person. Similarly, it helps to look at the details of your story one aspect at a time. While you're making these little changes, don't consider the effect on the piece as a whole - that will come later -



\begin{itemize}
\item Look at sentence length and paragraph length to see if they're too samey.
\item Look at word frequency. Are you over-using ``not'', ``still'', ``suddenly'' or ``but''?
\item Add imagery - don't worry about adding too much, you can always cut back later. E.g. ``There was a man whistling, walking along holding a can of Skol ahead of himself. He was holding the can like a compass'' (Ali Smith)
\item Look at each adjective and verb (change ``red'' to ``crimson''; ``walk'' to ``saunter''?).

\item Look at each minor character in turn, listing all they do

\item  If you want to look at each sentence in isolation,  replace full-stop by new-lines, or go through your piece from back to front.

\textbf{Exercise 3}

(Print the start of a story out, one sentence per slip of paper. Give each person a slip and get them to read the story out. Now get each of them to rewrite the sentence. Get them to read the revised story out) 

\end{itemize}




\subsection*{Medium scale}
These change the whole work, but only one feature of it. You can try something out as an experiment, to revive the story's freshness for you. Keeping the original, try
\begin{itemize}
\item Changing the location 
\item Changing the gender or age of a character
\item Changing the viewpoint
\item Changing the tense
\item Change the era (useful to solve the plot refutations involving the use of mobile phones, etc) 
\item Halving the word-count (it's interesting to see what parts really matter to you)
\end{itemize}
Even if you abandon these versions, trying them out may give you ways to add detail to what you've taken for granted. For example
\begin{itemize}
\item if you change the location to somewhere exotic you may mention the meals more. Why not mention meals in your original?
\item if you set it in an Arctic Research Station you may find that the confinement intensifies emotions. 
\end{itemize}

\subsection*{Large scale}
\textit{Re-writing from scratch} - produce a new draft of a chapter without looking at the old version! The differences between the versions might be revealing. 

 \textit{Outlining} - block-out your piece to see if it flows and if the proportions are ok - e.g. \textit{A page about Jim. A line about Mary. A page about London. How Jim ended up in London}. 

 \textit{Re-conceiving} - Sometimes when you re-read a story you might realise that the crux of the story isn't what you'd originally intended. Perhaps a secondary character has become more interesting than the main one. 

 \textit{Re-structuring} - Re-order chapters or sections. Are you starting too early?  You can put an important chapter at start, as a prologue, or put chapters in reverse. Instead of strictly chronological ordering of chapters, you could use alternate chapters according to location or point-of-view. Such tricks aren't especially avant-garde -
\begin{itemize}
\item  Iain Banks' "Use of Weapons" has 2 narrative threads - one going forwards in time, one going backwards.

\item Two recent best-sellers ("The Miniaturist" by Jessie Burton and  "Days without end" by Sebastian Barry) begin with funerals where the identity of the narrator doesn't become clear for pages, and the deceased's identity takes chapters to reveal (in "Days without end" we're told on p.62). Both first-chapters are flash-forwards. 

\item "Chang and Eng" by Darin Strauss is framed by the death-bed scene (the final chapter's a repeat of the first). The chapters alternate between 2 storylines: 1811-1842, and 1842-1874, 1842 being the year Chang and Eng met their wives. 
\end{itemize}

If you can't decide how to end a novel, why not include several options. Again, it's not avant-garde - in "Jane, Unlimited" (2017, it's YA) Kristin Cashore included 5 endings, each a different genre. 

\subsubsection*{Novels and short stories}
Maybe your novel could become some short stories. Maybe a part of a short story can immediately become a Flash piece. \url{http://cathyday.com/2012/02/novels-to-stories-stories-to-novel/} looks at how related short stories can be made into a novel and vice versa.

It's perhaps better to plan ahead if you're going to do this. One benefit of
writing a novel as a set of short stories is that you can start sending the 
stories off long before you've completed the novel - parts of ``A Visit from the Goon Squad'' by Jennifer Egan appeared initially in The New Yorker and Harper's. Jill Widner has published (and won prizes for) several story-like chapters of her novel that she hasn't yet published. Doing this also helps with marketing - a story can be a teaser for the novel.



\subsubsection*{Recycling}
Here's a piece of mine that I've had published -
\begin{narrow}{1.0cm}{1.0cm}
\begin{center}\textbf{Rejection}\end{center}

Suppose one person’s death could save the lives of many others by providing them with vital organs. Should the state intervene for the greater good?

In 2007 I wrote a story called “Going.” I entered it in a couple of competitions before deciding that no one wanted it. Its street market went into “Late” (published in By All Means), the tea flavours went into “Out” (Ink, Sweat and Tears), and the passage about hearing noises downstairs appeared in “Correspondence” (Necessary Fiction). None of those other stories were true, though “Going” was. Did I do the right thing?
\end{narrow}
There are 2 things to learn from this - firstly, you can publish the story of your failure; secondly, it's a true story - those parts
weren't wasted.

It's easy to flog dead horses, to work on a piece that you like but nobody else does. I've hung onto stories for years not realising that it's just one scene that I liked. You can cut your losses and recycle the bits (having of course saved the original version first). Pieces of dialogue, or even whole scenes can find new homes. 

 Bret Anthony Johnston’s ``Half of What Atlee Rouse Knows About Horses'' won the £30,000 Sunday Times EFG Short Story Award for 2017. The author said it was in a “process of accrual” for a decade.  He said he would be working on his novel and would get frustrated and leave it a while and write a little vignette about a horse.  When he started to have enough of the vignettes and his character became a “backbone” to them, he spread them over the floor and called it a story.  


I've not gone to that extreme, though I've had success sticking fragments together, putting ``* * *'' between them or numbering them.


\subsection*{See also}
\begin{itemize}
\item ``Self-Editing for Fiction Writers'' by Renni Browne and Dave King (The authors state that, since writing and editing are two different skills, you should not try to do both at once
\item ``Manuscript Makeover'' by Elizabeth Lyon
\item 
\url{http://www.bovary.fr/folio_visu.php?mode=sequence&folio=&org=3&zoom=50&seq=91} - Flaubert's drafts
\item \url{https://poetryschool.com/poems/how-to-renovate-a-morris-minor/} - Jonathan Edwards' description of a poem from first draft to publication
\item \url{https://www.nytimes.com/2017/08/01/books/review/corral-collins-zhang-poetry-works-in-progress.html} - Billy Collins, etc
\item \url{http://necessaryfiction.com/writerinres/AMonthofRevision} - \textit{Necessary Fiction}'s collection of articles
\item \url{http://emmadarwin.typepad.com/thisitchofwriting/resources.html} - Emma Darwin's articles (see REVISING, RE-VISITING, RE-ENVISIONING)
\item \url{https://stingingfly.org/2017/10/24/edit-lousy-writing/} - ``How To Edit Your Own Lousy Writing'' by Julian Gough 
\end{itemize}

\newpage
\section{Miscellaneous Exercises}

\includegraphics[width=5cm]{writerstoolbox.jpg}
Out of ideas? Fed up with soul-searching? Why not take a break and relax at an evening of fun and games. Without realizing it, you might go home with enough material to last the rest of the year. Bring writing materials!

Here are some exercises. There are some old favorites that never fail, and some that have never been used before and might never be used again. You'll be writing against the clock. Use that time pressure constructively 

Note that
\begin{itemize}
\item Deadlines and constraints help!
\item The exercises make you start in an unfamiliar place. There's a natural urge to return to familiar settings. The exercise exploits that urge
\item You're not expected to produce a finished work. Later you may decide to remove the scaffolding (i.e. the pretext of the exercise) or just use a fragment 
\item You can treat them as a limbering-up exercise - just throw them away and go on to the next one. 
\end{itemize}

We won't only have games. We'll borrow, recycle, scavenge - anything to get ideas on the cheap. Let's start with an easy one -

\begin{itemize}
\item Change LOVE to HATE a letter at a time. 
\end{itemize}

\subsection*{Games}
\subsubsection*{Card tricks}
The oldest trick in the book. Often works.

``Many writers, including me, often begin the idea of a whole novel with only a single image" - Elizabeth Baines. 

Deal out 2 postcards to each person.

%\subsection*{Poetic Justice}
%\includegraphics[width=5cm]{poeticjustice.jpg}

\subsubsection*{Misfits}

\includegraphics[width=5cm]{misfits.jpg}

Why, I wonder, did the name of this game remind me of writers? On 2 pieces of paper, get each person to write 2 distinctive, quirky traits of an individual. Collect the pieces of paper, shuffle them, deal them out, and get people to write a short scene using the fragments.

Note that you can emphasize a trait by juxtaposing its opposite. Holmes= rational + dreamer, druggie, violinist. The evil master strokes his cat.

Write some dialogue where one character make a point by saying the opposite of what they mean. Try not to make it merely sarcastic.


(from Geraghty) -  2 distinctive characters meet (think Dr Who) - Queen Victoria and Gandalf, Alice in Wonderland and Johnny Rotten. Choose a situation where 2 people meet


\begin{itemize}
\item Blind date
\item Knock in door collecting for charity;
\item complaints desk;
\item teacher wants to see pupil after school;
\item order a meal;
\item police flag someone down
\item going through customs
\item being shown around house
\end{itemize}

\subsection*{Recycling}

\subsubsection*{Bottles}
Queen Elizabeth I appointed an Uncorker of Ocean bottles. Capital offence for others to open them

``Imagine you're walking on the beach one warm sunny morning when you notice something half-buried in the sand. As you get closer, you see it's a bottle with something inside. You open the bottle'' (Geraghty)


\subsubsection*{Common family/partner anecdote}
The ones you dreaded when you brought a boy/girlfriend home to your parents for the first time. Or the ones you keep repeating. Why do they work? (punting priests; ``Together'')



\subsection*{Things}
\subsubsection*{Strange}
An unusual thing in a room. How did it get there? (explaining the unknown, returning to the familiar)


\subsubsection*{Interviewing an inanimate object}
As if it were a celebrity - Giza Sphinx; Holmes' violin; Excaliber; Swimming Pool, Nelson's eyepatch; Steve Hammond's computer. Snow White's mother's mirror? If you can't think of a good interview, at least come up with some interesting objects.



\subsection*{Games of Chance}


\subsubsection*{Edgar Wallace} - Plot wheel. Origami. 4 situations. 4 motives. Mix!
\begin{itemize}
\item wife cures husband of alcoholism - love
\item detective seeks murderer - justice
\item daughter gives up career to care - duty
\item a secretary lies to save boss (who she doesn't like) - revenge
\end{itemize}

\subsubsection*{People and Places}
4 famous people. 4 famous scenes. Mix!

Nelson Trafalgar. Bannister 4-minute mile. Sherpa/Tensing Everest
Parting of Red Sea


\subsubsection*{Order}
extract significant elements of story. Introduce them in a random order




\subsection*{Misc}
\subsubsection*{Missing the obvious}
Think of an obvious act of success. Then imagine a plot that involves deliberately trying to achieve the opposite - e.g. deliberately losing a race. 

\subsubsection*{Two column}
Draw a line down the middle of the page. Write the parallel thoughts of 2 people (on train?). Do this exercise in pairs?

\subsubsection*{Selfie}
Write a snapshot of yourself. How do others see you?  Put yourself somewhere interesting!

\subsubsection*{Odd Man Out PoV}
The only dwarf in the basketball team. 

The only pregnant woman at a funeral

The only man at a pre-natal workshop

\subsection*{Books}
\begin{itemize}
\item ``More five-minute writing'' - Margret Geraghty  - Constable and Robinson, 2013

\end{itemize}



\subsection*{In my end is my beginning}
These are the endings of famous novels. Use them to start a story.
\begin{itemize}
\item As he turned the corner Salvatore was coming into the courtyard to find his Vespa. He called out that he was going back to Florence and would be starting first in the morning for Rio-maggiore - Innocence, Penelope Fitzgerald

\item It is a far, far better thing that I do, than I have ever done; it is a far, far better rest than I go to than I have ever known - A Tale of Two Cities, Charles Dickens

\item The offing was barred by a black bank of clouds, and the tranquil waterway leading to the uttermost ends of the earth flowed sombre under an overcast sky – seemed to lead into the heart of an immense darkness. - Heart of Darkness, Joseph Conrad

\item I lingered round them, under that benign sky; watched the moths fluttering among the heath, and hare-bells; listened to the soft wind breathing through the grass; and wondered how any one could ever imagine unquiet slumbers for the sleepers in that quiet earth. - Wuthering Heights, Emily Bronte

\item He had done what he had wanted to do. He could now never go back to Leith, to Edinburgh, even to Scotland, ever again. There, he could not be anything other than he was. Now, free from them all, for good, he could be what he wanted to be. He’d stand or fall alone. This thought both terrified and excited him as he contemplated life in Amsterdam. - Trainspotting, Irvine Welsh

\item The eyes and faces all turned themselves towards me, and guiding myself by them, as by a magical thread, I stepped into the room - The Bell Jar, Sylvia Plath

\item The scar had not pained Harry for nineteen years. All was well - Harry Potter and the Deathly Hallows, JK Rowling 
\end{itemize}


\subsection*{proverb and idioms}
Either
\begin{itemize}
\item \textit{Write a story} by taking a proverb literally
\item \textit{Make a list} of updated idioms. Here are some that need updating -

A needle in a haystack.

A bird in the hand is worth two in the bush

You can't teach an old dog new  tricks
\end{itemize}

\subsection*{Definitions}
\begin{itemize}
\item Innuendo - an Italian suppository
\item Stockade - fizzy OXO. Like LiveAid?
\end{itemize}

\end{document}
